%
% IF
\ifthenelse {\boolean{thesis}}
%
% THEN
{
State of Charge (SoC) estimation is critical for battery management in any battery-based electrical system.
It acts as a criterion, which justifies the amount of energy left in a single or pack of cells and how long they can supply power, for example, the endurance of electric vehicles before subsequent charging.
The ability to estimate a correct SoC and manage consumption accordingly, not only to prolong the time between charges but also the general health of the battery, in the long run, became vital for any electrical grid.
The most common way of estimating batteries charge without direct physical intervention is battery modelling.
It is possible to simulate battery behaviour by defining specific constant parameters like internal resistance and capacitance and training over sensory data of battery voltage and current consumption.
However, while some methods reasonably estimate the charge percentage, the battery losses created by Coulombic efficiency can create unpredictable behaviour with different impacts.
The Machine Learning models tend to estimate parameters themselves through the long process of training based on statistical data of battery utilization, rather than constant definitions.
The Recurrent Neural Network (RNN) treats Coulombic efficiency as part of the system by considering a short period of samples rather than single readings.
Since the wide usage of Machine Learning in many applications, there have been multiple attempts to estimate the State of Charge using Neural Nets with battery sensory data, such as Voltage, Current and Temperature.
Those methods attempted multiple model training techniques to minimize the discrepancy in available readings caused by general sensory inaccuracy and develop the best possible SoC estimator, derived from Coulomb Counting with battery testing machine.
Although it does not match actual Electric Vehicle driving, the technique validation methodology makes a reasonable way of evaluating the most efficient and low-resource consuming method.
Through the long process of critical analyses, implementation, review and comparison of the already published RNN models over multiple drive cycles, the most promising methods ended up being a Stateless model with an approximate 5-minute long history of samples, trained on LSTM model with robust adaptive Online gradient learning algorithm (RoAdam) as optimizer.
The Root Mean Squared accuracy of such models tends to be around 3.64\% against the datasets, which were not involved in the training process.
All models went through a cross-validation process over multiple datasets, and some were tested on a few embedded solutions, which can be integrated into electrical circuitry of any type.
As a result of the research, several weaknesses have been identified during reviews, such as incorrect or inefficient weight distribution across input parameters and lack of accuracy as opposed to general RNN models in other scenarios.
}
%
% ELSE
{
% \begin{itemize}
%     \item Intro statement (Application of State of Charge or Batteries)
%     \item What SoC stands for?
%     \item What or how it has been done already?
%     \item In the present paper - Investigate best ML method as a replacement to already existing ones.
%     \item How is it going to be conducted. (2-3 sentences).
%     \item What was the final conclusion or finding. That neither of them are effective and further investigation is required to be certain of safety usage inside a car.
% \end{itemize}
State of Charge (SoC) estimation is a crucial property of a Lithium-Ion battery in any electrical system.
An accurate charge percentage, without any discrepancies, allows a battery management system (BMS) to enhance battery module utilisation further, allowing higher endurance or prolonging overall utilisation for more cycles.
There have been numerous attempts to determine SoC without physical intervention, such as battery modelling.
Since batteries commonly have non-linear characteristics, statistical-based Machine Learning methods have the potential to replace the traditional model.
The present paper conducts a practical usability investigation on Recurrent Neural Networks (RNN) to determine the best-suited machine learning method for Electric vehicles (EV).
Through the implementation of models from multiple published sources and cross-validation testing with several driving scenarios of Lithium-Ion batteries, several models have been assessed on accuracy and drawbacks.% against simulations of different driving conditions.
\textcolor{red}{Each method will be examined by applied techniques, highlighting implementation and attempts to implement from the available framework as close as possible to the source with available battery data.}
The results highlight the percentage accuracy in the average of 3.64\% of Root Mean Squared Error, several drawbacks of the overall implementation and propose potential solutions for further improvement.
Every implemented model had a similar drawback of poor capturing of the middle area of a charge, applying higher weight on a voltage rather than current.
The combined techniques in a single custom model could be better suited to improve accuracy further.
}