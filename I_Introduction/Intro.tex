%
% Importance of Electric Vehicles and their battery problem.
\IEEEPARstart{T}{he} market for Electrical Vehicles (EVs) has grown significantly over the past decade~\cite{state-ev-australia}.
The replacement of a fossil fuel-based engine with an electric drivetrain eliminates exhaust emissions with the potential to significantly reduce human impact on climate change.
For EVs to grow market share and reduce costs, battery cost and longevity must be improved.
Extensive battery cycling leads to battery degradation over time (ageing).
The development of smarter and more accurate battery management strategies may be capable of prolonging service duty.
This would rely upon a system's ability to estimate a battery's state at any point in time.
An accurate charge calculation avoids overcharging or over-discharging, leading to improved battery service utilisation, better health estimation, longer life span, more reliable range prediction and further benefits~\cite{calif_proper_2008}.

%
% The state of Charge, methods and its' value on BMS.
The development of effective methods for State-of-Charge (SoC) estimation remains a topic of crucial research focus.
Various techniques to estimate the SoC have been developed to enhance battery usage.
The ability to determine the state of a battery or a battery system is a required function for an advanced Battery Management System (BMS).
Those techniques can be classified into three primary categories~\cite{ali_towards_2019,ng_enhanced_2009,robust_SoC,6953745}: direct measurement, model-based methods, and computer intelligence or Machine Learning (ML).
Direct measurement methods take readings from batteries relying on sensors such as open circuit Voltage, internal resistance, or current readings over set periods (i.e. Coulomb Counting)~\cite{ng_enhanced_2009,robust_SoC}.
Model-based methods recreate battery behaviour and use sensor inputs to calculate results from a pre-defined model~\cite{6953745}.
Computer intelligence techniques enhance such models with additional data.
Those data-driven calculations aim to improve model estimation by fitting to an actual behaviour observation.
Examples include Fuzzy Logic~\cite{malkhandi_fuzzy_2006}, Support Vector Machine~\cite{hansen_support_2005, anton_battery_2013}, or Neural Networks (NN)~\cite{song_lithium-ion_2018,Chemali2017,mamo_long_2020,jiao_gru-rnn_2020,xiao_accurate_2019,javid_adaptive_2020,zhang_deep_2020}.

%
% Comparison between traditional methods and ML-based estimations.
While some model-based methods, such as the equivalent circuit model, are simple to implement within a BMS, many cannot correctly capture a battery's complex multi-dependant behaviour~\cite{6953745}.
%Others are limited to offline simulation due to high complexity and computation requirements, therefore, not suited to an onboard BMS.
Direct measurement estimation is limited to sensor accuracy and is affected by losses created by Coulombic efficiencies~\cite{Smith_2010} where some portion of charge gets transferred to heat or is affected by battery ageing that is not captured.
%Additionally, the measured voltage depends on whether the battery is in use or has been on rest for a given time.
%An accurate SoC estimation model has to implement a way to accommodate battery losses and sensor readings.
%If model-based methods treat as constant values, then in real scenarios, they can be significantly limited to sensor inaccuracy, environmental and internal battery temperature, or ageing.
%In contrast, Neural Network methods can accommodate losses as they can capture complex phenomenological behaviours~\cite{bengio_learning_1994}.
%
%Smart BMSs, which incorporate some method of ML, use only sensory data~\cite{zhang_deep_2020} and do not require batteries' physical property~\cite{zhang_deep_2020}.
In contrast, Machine Learning can establish relationships in complicated and multi-dimensional non-linear systems~\cite{hansen_support_2005,anton_battery_2013,he_state_2014}.
This characteristic shows excellent potential to account for battery losses due to Coulumbic efficiency.
Some researchers used the support vector machine-based methods to estimate SoC using voltage, current, and temperature inputs~\cite{hansen_support_2005,anton_battery_2013}.
Sensory data was obtained from a driving schedule profile on a battery cycler, and the end error estimation was achieved as less than 6\%~\cite{he_state_2014}.
%The solution to lowering that percentage may lie within the more complicated models, like Neural Networks.
Many attempts to implement different Neural Networks exist, but the most promising variant for charge estimation is the Recurrent Neural Networks (RNN)~\cite{song_lithium-ion_2018, Chemali2017, mamo_long_2020, jiao_gru-rnn_2020, xiao_accurate_2019, javid_adaptive_2020, zhang_deep_2020}.
The effectiveness of RNNs in time-series dependant problems has been shown using internal neurons to process data sequences with varying lengths by Chemali et al.~\cite{Chemali2017}.
% The models' cells act as memory units, building relationships and giving outputs based on multiple inputs over time.
% Typical examples are data forecasting, handwriting, speech and image recognition, machine translation or music composition~\cite{devdarshan_applications_2019}. 

% 
% What has been done in the area of RNN for battery characterisation
Over the past years, the RNN approach has found multiple applications for SoC estimation.
The earliest approach utilised a battery charge's regression nature only using stateless models~\cite{song_lithium-ion_2018,jiao_gru-rnn_2020,xiao_accurate_2019}.%\textcolor{red}{when input impact is not preserved for further predictions}.
% non-connected predictions, allowing to be used at any time, independently from previous results or inputs.
Later, some approaches introduced additional parameters to support the NN learning process~\cite{mamo_long_2020,jiao_gru-rnn_2020,javid_adaptive_2020}.
Besides good convergence, these models can determine critical events, like the time before complete charge depletion or overcharge.
However, the wide application has been limited due to the need for the initial state as an input feature.
The most popular approach determines the charge's value using a fixed size recent history of voltage, current, and temperature in stateless Long Short-Term Memory (LSTM) models~\cite{Chemali2017,mamo_long_2020,javid_adaptive_2020,zhang_deep_2020}.
This method has the advantage of being independent of the charge or discharge cycles at different periods as long as the history samples are in equally time-spaced order.
%However, estimation determines values based on fixed history samples and does not preserve for the next prediction, making practical estimation of single or several charge values, but not determining critical events ahead.
The most recent attempt to determine the Li-Ion battery's remaining useful life was with the implementation of the Gradient Recurrent Unit (GRU) models~\cite{song_lithium-ion_2018,javid_adaptive_2020,xiao_accurate_2019,jiao_gru-rnn_2020}, when every prediction independent from previous, allowing it to be used at any random point of time, without worrying if a battery was initially fully charged or depleted.
% An alternative method could be with stateful models, which continuously preserve the prediction's impact and can be propagated further until they reach the end state without resets~\cite{zhu_statefulnes_tfdocs_2020}.
While it applies to the estimation of regenerative braking, the stateful models are more applicable for a critical event time estimation, like a prediction of the remaining batteries' life.
% \mbox{Table~\ref{tab:review}} summarises the range of methods applied to SoC estimation, highlighting the model cell type, structure to define input sample type, optimiser selection, and additional features introduced by authors to improve predictions.
Focusing specifically on RNN models applied to SoC estimation, \mbox{Table~\ref{tab:review}} presents a range of such methods developed in recent years.

%
% Intro where are specifics. Hyper-para. THere is w\a hole range, Table summarises many.
The earliest attempts to train an RNN model to predict the SoC was to fit several cycles of a single battery utilisation dataset at different temperatures~\cite{song_lithium-ion_2018, xiao_accurate_2019,javid_adaptive_2020, jiao_gru-rnn_2020}.
Later, to generalise battery behaviour to multiple usage scenarios, leading to higher Root Mean Squared Error, but broader application, like in Mamo \textit{et al.}~\cite{mamo_long_2020} work.
%For instance, by comparing similar procedures of testing from Song et al.~\cite{song_lithium-ion_2018} and Mamo et al.~\cite{mamo_long_2020} who conducted their research methods but used different validation mechanisms, it can be seen how accuracy error doubles if testing performs at a different temperature or untrained driving profile.
This approach led to doubled accuracy error on the testing data, perfomed at a different temperature or untrained driving profile, as compared with similar testing procedures from Song et al.~\cite{song_lithium-ion_2018} and Mamo et al.~\cite{mamo_long_2020}.
% By selecting the same drive cycles at a closely similar temperature, Song et al.~\cite{song_lithium-ion_2018} accuracy achieved roughly 0.735\% error, then as Mamo et al.~\cite{mamo_long_2020} reported 1.2533\% at best.
Doubling the amount of data by combining several temperatures or profiles also showed insignificantly higher error but improved general capture, as per stateful models in Song \textit{et al.}~\cite{song_lithium-ion_2018} with roughly 0.735\%, and stateless \textit{Mamo et al.}~\cite{mamo_long_2020} reporting 1.2533\%  errors respectively.
Those numbers can be explained by using a single driving profile but with an entire available temperature range for training and then a portion of handpicked temperatures for validation and accuracy reporting.
Such an approach does not necessarily represent a realistic EV usage scenario well since, during a single acceleration event, the battery goes through ambient temperature to the maximum allowed within several seconds, and its' usage depends on the road conditions and the driver's experience.
One of the potential to improve that capture is to modify the structure of the models, introducing an additional layer of logic, like Attention, as per Mamo et al.~\cite{mamo_long_2020}, or Extra Dense layers, as per Jiao et al.~\cite{jiao_gru-rnn_2020} making a model applicable to any driving condition.
% There has been little research to validate the performance of different Machine Learning techniques to extrapolate ideal laboratory battery cycling conditions of early collected data, to an electric vehicle behaviour.
%% (Include one of the sections)
% After identifying the most optimal input preferences without significant effect on the performance, one of the further developments applies changes to the structure of GRU or LSTM by adding additional layers (Attention or Extra Dense layers)~\cite{mamo_long_2020, jiao_gru-rnn_2020}.
% Another way is to change the optimisation process to achieve similar accuracy faster (i. e., adding a Momentum algorithm to the Stochastic Gradient Optimisations process or replacing gradient calculations with statistical)~\cite{xiao_accurate_2019, javid_adaptive_2020}.
% Those methods modify standard ways introduced earlier in model training by applying additional operations.
% As a result, they met to achieve faster accuracy using similar training approaches. 
Another strategy would be is to use a variety of statistical or gradient-based optimisers (i. e., adding a Momentum algorithm to the Stochastic Gradient Optimisations process~\cite{xiao_accurate_2019}) to speed up the training or extra multiple potential minimal, achieving the lowest error or identifying the most suited for a given scenario.
Due to the stochastic nature of ML, it is hard to give any clear winner among optimisers by only judging their complexity, not average performance with multiple trials.
% \textcolor{red}{\textbf{Matt: I moved most of the details about the table to caption. However, since this paragraph and the next were swapped, I had to do my best to avoid driving profile mentioning where possible. Perhaps we should consider swapping them back or some sentences.}}
\begin{table*}[h]
    \renewcommand{\arraystretch}{1.3}
    \caption{Evaluated papers implementation summary.
    The model type highlights a primary path in structuring a Neural Network.
    Statefulness defines the input method, where stateless uses a fixed size of input samples per feature and statefully applies each time-sample one at a time in batches.
    %Optimiser selection sets the algorithm for the learning process from one of the following methods: Adaptive moment estimation (Adam), Nesterov adaptive moment estimation  (Nadam), Stochastic gradient descent (SGD), AdaMax (AM) and Differential Evolution (DE).
    Optimisers are defined from Adaptive moment estimation (Adam), Nesterov adaptive moment estimation  (Nadam), Stochastic gradient descent (SGD), AdaMax (AM) and Differential Evolution (DE).
    }
    \centering
    \label{tab:review}
\resizebox{\textwidth}{!}{
\begin{tabular}{l|c|c|c|c|c|c|c|c|c|l}
    \hline\hline \\[-4mm]
    \multicolumn{1}{ c }{} & 
    \multicolumn{2}{|c|}{Model} & 
    \multicolumn{2}{ c|}{State-} & 
    \multicolumn{5}{ c|}{Optimiser} &
    \multirow{2}{ 4em }{Extension} \\
    \cline{1-4} \cline{5-10}
    Reference source & GRU  & LSTM & -less & -ful & Adam & Nadam & SGD & AdaMax & DE\footnote{Differential Evolution} & \\
    \hline
    Song~\cite{song_lithium-ion_2018}
        & \chk &      &      & \chk & \chk &      &      &      &      & 4 Layers  \\
    Chemali~\cite{Chemali2017}
        &      & \chk & \chk &      & \chk &      &      &      &      &           \\
    Mamo~\cite{mamo_long_2020}
        &      & \chk & \chk &      &      &      &      &      & \chk & Attention \\
    Jiao~\cite{jiao_gru-rnn_2020}
        & \chk &      &      & \chk &      &      & \chk &      &      & Momentum  \\
    Xiao~\cite{xiao_accurate_2019}
        & \chk &      &      & \chk &      & \chk &      & \chk &      & Ensemble  \\
    Javid~\cite{javid_adaptive_2020}
        & \chk &      & \chk &      & \chk &      &      &      &      & Robust    \\
    Zhang~\cite{zhang_deep_2020}
        &      & \chk & \chk &      &      & \chk &      &      &      & Online    \\
    \hline\hline
\end{tabular}
}
\end{table*}

% Although experiments which involved extrapolation of ideal laboratory conditions with accurate sensory to real-world conditions, worth different environmental characteristics or sensory interference - are not currently available.
%
In most published testing of ML methods applied to SoC, experiments on battery cycling data were conducted on different cell types.
Most used table data of real-time sensory results from battery cyclers to validate efficiencies generated using different current schedulers (driving profiles)~\cite{Chemali2017,song_lithium-ion_2018,mamo_long_2020,xiao_accurate_2019}.
%An equally time-based sample of current consumption acts as an input to the battery cycler, intended to recreate a stress test on a battery or human driving behaviour.
Three profiles are most commonly used in the research in this area: Dynamic Stress Test (DST) for a variable power discharge mode, aggressive Highway Drive Schedule (US06) and Federal-Urban Driving Scheduler (FUDS) for nominal driving scenarios~\cite{xiao_accurate_2019,javid_adaptive_2020,mamo_long_2020}.
Unlike some general simple static discharge processes, which commonly appear in other battery-based tools, driving profiles include some amount of regenerative driving to simulate the actual application of the battery for an electric vehicle.
Different of such drive cycle data in training and testing of Machine Learning SoC estimation have been used including applications focusing the fitting process on battery discharge only~\cite{song_lithium-ion_2018,mamo_long_2020,jiao_gru-rnn_2020,javid_adaptive_2020}; capturing the complete charge-discharge cycle~\cite{Chemali2017}; multiple combinations at various temperatures or profiles~\cite{xiao_accurate_2019,mamo_long_2020,Chemali2017,javid_adaptive_2020}; the impact of data samples ammount~\cite{song_lithium-ion_2018}; and cross-validation of all three current profiles against each other~\cite{mamo_long_2020}.
% There were several applications for those data, involving focusing the fitting process on battery discharge only~\cite{song_lithium-ion_2018,mamo_long_2020,jiao_gru-rnn_2020,javid_adaptive_2020}, capturing complete charge-discharge cycle~\cite{Chemali2017}, multiple combinations at various temperatures or profiles~\cite{xiao_accurate_2019,mamo_long_2020,Chemali2017,javid_adaptive_2020}, the impact of data samples ammount~\cite{song_lithium-ion_2018} and cross-validation of all three current profiles against each other~\cite{mamo_long_2020}.
%
% Mamo et al.~\cite{mamo_long_2020} conducted experiments by validating the RNN model performance by training one driving profile and testing against the other two.
% The results showed doubled higher Root Mean Squared Error, as opposed to training and validation over single driving profiles as per experiments by Xiao et al.~\cite{xiao_accurate_2019}.
%
% \textcolor{red}{Follow the enumaeration}
% \begin{enumerate}
%     \item only discharge~\cite{song_lithium-ion_2018,mamo_long_2020,jiao_gru-rnn_2020,javid_adaptive_2020}
%     \item charge-discharge~\cite{Chemali2017}
%     \item multiple of cycles, like temps or profiles~\cite{xiao_accurate_2019,mamo_long_2020,Chemali2017,javid_adaptive_2020}
%     \item Limits of one cycle, benefits of multiple or length of it,  examples to compare to~\cite{mamo_long_2020,song_lithium-ion_2018}
% \end{enumerate}
Identifying the best-suited method for a specific condition, like driving an EV, is one of the crucial steps for machine learning engineering.
It requires a carefully defined methodology, which characterises researched conditions as close as possible, and experimental results from multiple models with applicable techniques and the lowest errors.
By comparing implementation and results from different sources and testing accuracy and performance against multiple driving conditions at various temperatures from ambient to maximum possible, it is possible to select the best machine learning technique, which can be integrated directly into an Electric vehicle and safely used either on tight city roads or long high-speed highways.
%There have not been many similar experiments against other SoC estimation methods since such validation procedures are necessary for the computer intelligence method compared to battery modelling.
% In addition, all investigations did poor research in extrapolating ideal laboratory conditions with battery cycler, stable temperatures and no sensor discrepancies to an actual driving situation on the road, mainly due to lack of car data.
%
% Usage of a single profile's battery cycling data may not validate ML methods' efficiency in driving an electric vehicle.
% The inability to determine the charge's current state during EV driving makes the online learning process inapplicable.
% Even with other SoC estimation technique usage, the computational complexity of training any NN is complicated to fit on a low-power device.
% An offline-trained model had the advantage of insignificant resource consumption during the prediction stage, making it a prefered way for an EV.
% All further model testing will be applied through varying battery cycling profiles to capture the influence of data type to model efficiency.
%
% (No point)
% The drawback lies within the validation procedures of produced models.
% Ideally, this process must be performed with similar data but under different conditions or at different times.
% However, once the produced model is taken through different unseen scenarios, the accuracy lowers drastically compared to early reported results since it starts to experience something completely unexpected.
% The lack of training samples or means to produce at different conditions affected the final judgment results of many publications. 
% Even though the estimation produces accurate output with tabled data, placed under actual driving conditions - the model will have difficulties matching training performance. 


%
%
% Mamo et al. conducted experiments
%
% ------------------------------------------------------------------------ \\
% PAPER CONTRIBUTION may be worth adding as a clear statement \\
% ------------------------------------------------------------------------ \\
% Asses handpicked articles from different categories of SoC estimation, asses efficiency and complexity to utilise in EV.
% This paper's contribution lies in researching recent approaches used to estimate Charge's State from battery sensor readings.
% Comparing and implementing the most promising algorithms intended to determine the direction to build upon in the future. \\ \\
% This paper will test recent developments in the most common SoC prediction to determine the most promising direction/approach towards implementing an improved light model applied to an electric vehicle.
% It determined to find the ML model creation's weakness and implement a new one for an EV. \\
% ------------------------------------------------------------------------ \\
%
%
% This paper investigates, implements, and compares extended memory-based models of RNN to predict the State of Charge and additional built-on over-time techniques to select the most effective and least resource-requiring onboard-based computations applicable within Electrical Vehicles.
This paper investigates, implements, and compares extended memory-based models of RNN to predict the State of Charge and additional built-on over-time techniques to select the most suitable practical application for EV use with combinations of different profiles.
Each subset will contain implementation from various key references changing either structure of the models or learning approaches.
It should help develop a methodology to further extrapolate offline trained methods from the lab condition to the road drive tests.
% The recent advancement and commonly used subsets are Gradient Recurrent Unit and Long Shot-Term Memory unit cells.
% Recurrent Neural Network has been confirmed suitable for the battery-related system by authors discussed further, such as Chemali~\cite{LSTM_Hochreiter1997}.
% However, there is no valid proof of if GRU or LSTM is helpful for battery SoC estimation.
% The best way to separate them and explore differences and efficiencies is to use them as Stateful and Stateless models.
The A123 Lithium-Ion battery data with three typical driving profiles, obtained from the University of Maryland 2012~\cite{noauthor_calce_2017} cycling experiment, will act as training and testing samples.
Each method will be validated through those samples (either DST, US06 or FUDS driving profiles) and tested against the robustness and accuracy of estimation of the State of Charge of batteries in the other two unseen schedulers.

Since there has been no comparison between which RNN type or driving profile impacts State of Charge estimation for both charge and discharge cycles, this article aims to identify the most viable and optimum method for custom build Electric vehicles.
% Sometimes, multiple published models presented results relying only on discharge cycles, describing the efficiency of modelling on a single battery or in an ideal lab environment~\cite{xia_state_2018,javid_adaptive_2020,jiao_gru-rnn_2020,mamo_long_2020,zhang_deep_2020}.
However, long overnight charge cycles and regenerative breaking burst charges are equally crucial for the SoC percentage in the context of Electric vehicles' battery utilisation at prolonged usage and influence models' weight and biasses.
% \textcolor{red}{\textit{
This article will focus on the accuracy of SoC prediction based on the model training from charge and discharge cycles across various temperatures.
% (2)This article will present a novel methodology for comparing multiple models with different techniques to each other, accommodating MLs' stochastic nature.
% (3)This article will compare multiple works in SoC estimation to identify the most applicable to car driving and prose methods to build on top further.
% }}
% ~\cite{Chemali2017,}
%
%
The remaining sections are organised as follows.
Details on algorithms and optimisers are written in Section~\ref{sec:Body}, where Subsection~\ref{subsec:structure} separates all details for each GRU and LSTM method, and Subsection~\ref{subsec:optimisers} breaks down every applied optimising algorithm.
Applied methodology with details regarding training procedure and hyper-parameters selection are outlined in Section~\ref{sec:Meth}, with processing data in Subsection~\ref{subsec:b_data}.
Section~\ref{sec:conclussion} gives the results of implementation and performance characteristics and concludes the critical analysis.
