\ifthenelse{\boolean{thesis}}{
    Following the Literature Review, this chapter will analyse the selected branch of Neural Networks in detail.
    It explores the Recurrent NN in depth, and establishes the methodology for further research and implementation 
} {
    %
    % Importance of Electric Vehicles and their battery problem.
    The market for electrical vehicles (EVs) has grown significantly in recent decades~\cite{state-ev-australia}.
    The replacement of fossil-fuel-based engines with electric drivetrains eliminates the exhaust emissions, with the potential to significantly reduce human impact on climate change.
    To increase the market share and reduce the costs of EVs, the batteries' cost and longevity must be improved.
    Extensive battery cycling leads to battery degradation over time (ageing).
    The development of smarter and more accurate battery management strategies may prolong the service duty cycle.
    This would depend on the system's ability to estimate the battery's state of charge at any point in time.
    An accurate charge calculation would avoid the occurrence of overcharging or overdischarging, leading to improved battery service utilisation, a better health estimation, a longer lifespan, a more reliable range prediction, and other benefits~\cite{calif_proper_2008}.
}

% The state of Charge, methods and its' value on BMS.
%
\ifthenelse{\boolean{thesis}}{
    The development of effective methods for state-of-charge (SoC) estimation remains a crucial research focus.
    Various techniques to estimate the SoC have been developed to enhance battery usage.
    The ability to determine the state of charge of a battery or a battery system is a required function of an advanced battery management system (BMS).
    Those techniques can be classified into three primary categories~\cite{ali_towards_2019,ng_enhanced_2009,robust_SoC,6953745}: direct measurement, model-based methods, and computer intelligence or machine learning (ML).
    Direct measurement methods take readings from the batteries, relying on sensors, such as the open-circuit voltage, internal resistance, or current readings over a set period (i.e., Coulomb counting)~\cite{ng_enhanced_2009,robust_SoC}.
} {
    \textls[-5]{The development of effective methods for state-of-charge (SoC) estimation remains a crucial research focus.
    Various techniques to estimate the SoC have been developed to enhance battery usage.
    The ability to determine the state of charge of a battery or a battery system is a required function of an advanced battery management system (BMS).
    Those techniques can be classified into three primary categories~\cite{ali_towards_2019,ng_enhanced_2009,robust_SoC,6953745}: direct measurement, model-based methods, and computer intelligence or machine learning (ML).
    Direct measurement methods take readings from the batteries, relying on sensors, such as the open-circuit voltage, internal resistance, or current readings over a set period (i.e., Coulomb counting)~\cite{ng_enhanced_2009,robust_SoC}.}
}

Model-based methods recreate the battery behaviour and use the sensor inputs to calculate the results using a predefined model~\cite{6953745}.
Computer intelligence techniques enhance such models with additional data.
These data-driven calculations aim to improve the model estimation by fitting it to the actual observed behaviour.
Examples include fuzzy logic~\cite{malkhandi_fuzzy_2006}, support vector machines~\cite{hansen_support_2005, anton_battery_2013}, or neural networks (NN)~\cite{song_lithium-ion_2018,Chemali2017,mamo_long_2020,jiao_gru-rnn_2020,xiao_accurate_2019,javid_adaptive_2020,zhang_deep_2020}.

% Comparison between traditional methods and ML-based estimations.
%
While some model-based methods, such as the equivalent circuit model, are simple to implement within a BMS, many cannot correctly capture a battery's complex, multiple dependencies~\cite{6953745}.
%Others are limited to offline simulation due to high complexity and computation requirements, therefore, not suited to an onboard BMS.
Direct measurement estimation is limited to sensor accuracy and is affected by the losses created by Coulombic efficiencies~\cite{Smith_2010}, where some portion of the charge is transformed into heat or is affected by uncaptured battery ageing.
%Additionally, the measured voltage depends on whether the battery is in use or has been on rest for a given time.
%An accurate SoC estimation model has to implement a way to accommodate battery losses and sensor readings.
%If model-based methods treat as constant values, then in real scenarios, they can be significantly limited to sensor inaccuracy, environmental and internal battery temperature, or ageing.
%In contrast, Neural Network methods can accommodate losses as they can capture complex phenomenological behaviours~\cite{bengio_learning_1994}.
%
%Smart BMSs, which incorporate some method of ML, use only sensory data~\cite{zhang_deep_2020} and do not require batteries' physical property~\cite{zhang_deep_2020}.
In contrast, machine learning can establish the relationships in complicated and multidimensional nonlinear systems~\cite{hansen_support_2005,anton_battery_2013,he_state_2014}.
This characteristic shows excellent potential to account for battery losses due to Coulombic efficiency.
Some researchers have used support-vector-machine-based methods to estimate the SoC using the voltage, current, and temperature inputs~\cite{hansen_support_2005,anton_battery_2013}.
Sensor data were obtained from a driving schedule profile on a battery cycler, and the achieved end-error estimation was less than 6\% in~\cite{he_state_2014}.
%The solution to lowering that percentage may lie within the more complicated models, like Neural Networks.
Many attempts to implement different neural networks exist, but the most promising variant for charge estimation is recurrent neural networks (RNNs)~\cite{song_lithium-ion_2018, Chemali2017, mamo_long_2020, jiao_gru-rnn_2020, xiao_accurate_2019, javid_adaptive_2020, zhang_deep_2020}.
The effectiveness of RNNs in time-series-dependent problems was shown using internal neurons to process the data sequences with varying lengths by Chemali et al.~\cite{Chemali2017}.
% The models' cells act as memory units, building relationships and giving outputs based on multiple inputs over time.
% Typical examples are data forecasting, handwriting, speech and image recognition, machine translation or music composition~\cite{devdarshan_applications_2019}. 

% What has been done in the area of RNN for battery characterisation
% 
In the last five years, the RNN approach has found multiple applications in SoC estimation.
The earliest approach utilised the regression nature of the battery's charging, only using stateless models~\cite{song_lithium-ion_2018,jiao_gru-rnn_2020,xiao_accurate_2019}.
Later, some approaches introduced additional parameters to support the NN learning process~\cite{mamo_long_2020,jiao_gru-rnn_2020,javid_adaptive_2020}.
%\textcolor{red}{when input impact is not preserved for further predictions}.
% non-connected predictions, allowing to be used at any time, independently from previous results or inputs.
In addition to good convergence, these models can determine critical events, such as the time before complete charge depletion or overcharge.
However, their wider application has been limited due to the need for the initial state as an input feature.
The most popular approach determines the charge's value using the recent history at a fixed voltage, current, and temperature in stateless long short-term memory (LSTM) models~\cite{Chemali2017,mamo_long_2020,javid_adaptive_2020,zhang_deep_2020}.
This method has the advantage of being independent of the charge or discharge cycles in different periods, as long as the historical samples are in an equally spaced order of time.
%However, estimation determines values based on fixed history samples and does not preserve for the next prediction, making practical estimation of single or several charge values, but not determining critical events ahead.
The most recent attempt to determine the Li-ion battery's remaining useful life  implemented gradient recurrent unit (GRU) models~\cite{song_lithium-ion_2018,javid_adaptive_2020,xiao_accurate_2019,jiao_gru-rnn_2020}, where every prediction was independent of the previous prediction, allowing for it to be used at any random point of time, without worrying about whether the battery was initially fully charged or depleted.
% An alternative method could be with stateful models, which continuously preserve the prediction's impact and can be propagated further until they reach the end state without resets~\cite{zhu_statefulnes_tfdocs_2020}.
While this applies to the estimation of regenerative braking, stateful models are more applicable to a critical event time estimation, such as the prediction of the remaining battery life.
% \mbox{Table~\ref{tab:review}} summarises the range of methods applied to SoC estimation, highlighting the model cell type, structure to define input sample type, optimiser selection, and additional features introduced by authors to improve predictions.
Focusing specifically on RNN models applied to SoC estimation, \mbox{Table~\ref{tab:review}} presents a range of  methods that have been developed in recent years.

% Intro where are specifics. Hyper-para. THere is w\a hole range, Table summarises many.
%
The earliest attempts to train an RNN model to predict the SoC aimed to fit several cycles of a single battery utilisation dataset at different temperatures~\cite{song_lithium-ion_2018, xiao_accurate_2019,javid_adaptive_2020, jiao_gru-rnn_2020}.
Later, this was used to generalise the battery behaviour to multiple usage scenarios, leading to a higher root-mean-squared error (RMSE) and broader applications, as in Mamo and Wang's~\cite{mamo_long_2020} work.
%For instance, by comparing similar procedures of testing from Song et al.~\cite{song_lithium-ion_2018} and Mamo et al.~\cite{mamo_long_2020} who conducted their research methods but used different validation mechanisms, it can be seen how accuracy error doubles if testing performs at a different temperature or untrained driving profile.
This approach led to doubled accuracy errors on the testing data when performed at a different temperature or on an untrained driving profile, as compared with similar testing procedures presented by Song et al.~\cite{song_lithium-ion_2018} and Mamo and Wang~\cite{mamo_long_2020}.
% By selecting the same drive cycles at a closely similar temperature, Song et al.~\cite{song_lithium-ion_2018} accuracy achieved roughly 0.735\% error, then as Mamo et al.~\cite{mamo_long_2020} reported 1.2533\% at best.
Doubling the quantity of data by combining several temperatures or profiles also led to insignificantly higher errors, but improved the general capture, as per the stateful models in Song et al.~\cite{song_lithium-ion_2018}, with roughly 0.735\%, and the stateless models in Mamo and Wang~\cite{mamo_long_2020}, reporting a 1.2533\% error, respectively.
These numbers can be explained by the use of a single driving profile; when using the entire available temperature range for training, a portion of hand-picked temperatures can be used to report on the validation and accuracy. %please check intended meaning has been retained. %? Sadykov: Complex and confusing, I would suggest to use the original structure of sentence but with corrected words.
Such an approach does not necessarily represent a realistic EV usage scenario, since during a single acceleration event, the battery can go from ambient temperature to the maximum allowed temperature within a few seconds, and its usage depends on the road conditions and the driver's behaviour.
One of the potential ways to improve this capture is to modify the structure of the models, introducing an additional layer of logic, such as attention, as per Mamo and Wang~\cite{mamo_long_2020}, or extra 'dense' layers, as per Jiao et al.~\cite{jiao_gru-rnn_2020}, making the model applicable to any driving conditions.
% There has been little research to validate the performance of different Machine Learning techniques to extrapolate ideal laboratory battery cycling conditions of early collected data, to an electric vehicle behaviour.
%% (Include one of the sections)
% After identifying the most optimal input preferences without significant effect on the performance, one of the further developments applies changes to the structure of GRU or LSTM by adding additional layers (Attention or Extra Dense layers)~\cite{mamo_long_2020, jiao_gru-rnn_2020}.
% Another way is to change the optimisation process to achieve similar accuracy faster (i. e., adding a Momentum algorithm to the Stochastic Gradient Optimisations process or replacing gradient calculations with statistical)~\cite{xiao_accurate_2019, javid_adaptive_2020}.
% Those methods modify standard ways introduced earlier in model training by applying additional operations.
% As a result, they met to achieve faster accuracy using similar training approaches. 
Another strategy would be to use a variety of statistical or gradient-based optimisers (i.e., adding a momentum algorithm to the stochastic gradient optimisation process~\cite{xiao_accurate_2019}) to speed up the training and explore multiple-potential minima, which could achieve the fewest possible  errors or identify the model that is most suited to a given scenario.
Due to the stochastic nature of ML, it is hard to present any clear winner among the existing optimisers by only judging their complexity, not their average performance over multiple trials.
% \textcolor{red}{\textbf{Matt: I moved most of the details about the table to caption. However, since this paragraph and the next were swapped, I had to do my best to avoid driving profile mentioning where possible. Perhaps we should consider swapping them back or some sentences.}}
\ifthenelse{\boolean{thesis}}{
    \begin{table}[h]
        \renewcommand{\arraystretch}{1.3}
        \caption{Evaluated papers implementation summary.
        The model type highlights a primary path in structuring a Neural Network.
        Statefulness defines the input method, where stateless uses a fixed size of input samples per feature and statefully applies each time-sample one at a time in batches.
        %Optimiser selection sets the algorithm for the learning process from one of the following methods: Adaptive moment estimation (Adam), Nesterov adaptive moment estimation  (Nadam), Stochastic gradient descent (SGD), AdaMax (AM) and Differential Evolution (DE).
        Optimisers are defined from Adaptive moment estimation (Adam), Nesterov adaptive moment estimation  (Nadam), Stochastic gradient descent (SGD), AdaMax (AM) and Differential Evolution (DE).
        }
        \centering
        \label{tab:review}
    \resizebox{\textwidth}{!}{
    \begin{tabular}{l|c|c|c|c|c|c|c|c|c|l}
        \hline\hline \\[-4mm]
        \multicolumn{1}{ c }{} & 
        \multicolumn{2}{|c|}{Model} & 
        \multicolumn{2}{ c|}{State-} & 
        \multicolumn{5}{ c|}{Optimiser} &
        \multirow{2}{ 4em }{Extension} \\
        \cline{1-4} \cline{5-10}
        Reference source & GRU  & LSTM & -less & -ful & Adam & Nadam & SGD & AdaMax & DE\footnote{Differential Evolution} & \\
        \hline
        Song \textit{et al.}~\cite{song_lithium-ion_2018}
            & \chk &      &      & \chk & \chk &      &      &      &      & 4 Layers  \\
        Chemali \textit{et al.}~\cite{Chemali2017}
            &      & \chk & \chk &      & \chk &      &      &      &      &           \\
        Mamo and Wang~\cite{mamo_long_2020}
            &      & \chk & \chk &      &      &      &      &      & \chk & Attention \\
        Jiao \textit{et al.}~\cite{jiao_gru-rnn_2020}
            & \chk &      &      & \chk &      &      & \chk &      &      & Momentum  \\
        Xiao \textit{et al.}~\cite{xiao_accurate_2019}
            & \chk &      &      & \chk &      & \chk &      & \chk &      & Ensemble  \\
        Javid \textit{et al.}~\cite{javid_adaptive_2020}
            & \chk &      & \chk &      & \chk &      &      &      &      & Robust    \\
        Zhang \textit{et al.}~\cite{zhang_deep_2020}
            &      & \chk & \chk &      &      & \chk &      &      &      & Online    \\
        \hline\hline
    \end{tabular}
    }
    \end{table}
} {
    \begin{table}[H]
        \renewcommand{\arraystretch}{1.3}
        \caption{Summary of evaluated %MDPI: please confirm whether the Vertical line can be removed.
        papers' implementation. %!Sadykov: From my trials, removing vertical lines makes the table unreadable, so I would advice keeping them. The similar thing can be said about Table 8, but the simplicity of the structure has allowed it. However, if you believe that it will not affect the understanding of the content, you are welcome to remove it.
        The model type highlights a primary path to structuring a neural network.
        Statefulness defines the input method, whereas stateless models use a fixed size of input samples per feature and statefully apply each time-sample individually, in batches.
        Optimisers are defined using adaptive moment estimation (Adam), Nesterov adaptive moment estimation (Nadam), Stochastic gradient descent (SGD), AdaMax (AM) and Differential Evolution (DE).
        }
        % \centering
        \label{tab:review}
        \begin{adjustwidth}{-\extralength}{0cm}
        \newcolumntype{C}{>{\centering\arraybackslash}X}
        \begin{tabularx}{\fulllength}{l|C|C|C|C|C|C|C|>{\centering}m{1.5cm}|C|l}
            % \hline\hline \\[-5mm]
            \noalign{\hrule height 1pt}
            \multirow{2}{*}{\textbf{Reference Source}\vspace{-6pt}}&
            \multicolumn{2}{c|}{\textbf{Model}} &
            \multicolumn{2}{c|}{\textbf{State}} &
            \multicolumn{5}{c|}{\textbf{Optimiser}} &
            \multirow{2}{*}{\textbf{Extension}\vspace{-6pt}} \\
            \cline{2-10}
            & \textbf{GRU} & \textbf{LSTM} & \textbf{-less} & \textbf{-ful} & \textbf{Adam} & \textbf{Nadam} & \textbf{SGD} & \textbf{AdaMax} & \textbf{DE \textsuperscript{1}} & \\
            \noalign{\hrule height 0.5pt}
            Song et al.~\cite{song_lithium-ion_2018}
            & \chk &   &   & \chk & \chk &   &   &   &   & 4 Layers \\\noalign{\hrule height 0.5pt}
            Chemali et al.~\cite{Chemali2017}
            &   & \chk & \chk &   & \chk &   &   &   &   &     \\\noalign{\hrule height 0.5pt}
            Mamo and Wang~\cite{mamo_long_2020}
            &   & \chk & \chk &   &   &   &   &   & \chk & Attention \\\noalign{\hrule height 0.5pt}
            Jiao et al.~\cite{jiao_gru-rnn_2020}
            & \chk &   &   & \chk &   &   & \chk &   &   & Momentum \\\noalign{\hrule height 0.5pt}
            Xiao et al.~\cite{xiao_accurate_2019}
            & \chk &   &   & \chk &   & \chk &   & \chk &   & Ensemble \\\noalign{\hrule height 0.5pt}
            Javid et al.~\cite{javid_adaptive_2020}
            & \chk &   & \chk &   & \chk &   &   &   &   & Robust  \\\noalign{\hrule height 0.5pt}
            Zhang et al.~\cite{zhang_deep_2020}
            &   & \chk & \chk &   &   & \chk &   &   &   & Online  \\
            % \hline\hline
            \noalign{\hrule height 1pt}
        \end{tabularx}
        \end{adjustwidth}
        {\footnotesize{\textsuperscript{1} Differential evolution}.}
    \end{table}
}

% Although experiments which involved extrapolation of ideal laboratory conditions with accurate sensory to real-world conditions, worth different environmental characteristics or sensory interference - are not currently available.
%
In most published testing of ML methods applied to SoC, experiments on battery cycling data are conducted on different cell types.
The most-used table data for real-time sensors are derived from battery cyclers to validate the efficiencies generated using different current schedules (driving profiles)~\cite{Chemali2017,song_lithium-ion_2018,mamo_long_2020,xiao_accurate_2019}.
%An equally time-based sample of current consumption acts as an input to the battery cycler, intended to recreate a stress test on a battery or human driving behaviour.
Three profiles are most commonly used in the research in this area: the dynamic stress test (DST), which is used for a variable power discharge mode, the aggressive highway drive schedule (US06), and the federal-urban driving scheduler (FUDS), for nominal driving scenarios~\cite{xiao_accurate_2019,javid_adaptive_2020,mamo_long_2020}.
Unlike some general, simple static discharge processes, which commonly appear in other battery-based tools, driving profiles include some amount of regenerative driving to simulate the actual battery application in an electric vehicle.
Differences in these drive-cycle data in the training and testing of machine learning SoC estimation have been highlighted, including in applications focusing on the fitting process of battery discharge~\cite{song_lithium-ion_2018,mamo_long_2020,jiao_gru-rnn_2020,javid_adaptive_2020}, capturing the complete charge--discharge cycle~\cite{Chemali2017}, multiple combinations at various temperatures or profiles~\cite{xiao_accurate_2019,mamo_long_2020,Chemali2017,javid_adaptive_2020}; the impact of data samples' quantity~\cite{song_lithium-ion_2018}, and the cross-validation of all three current profiles against each other~\cite{mamo_long_2020}.
% There were several applications for those data, involving focusing the fitting process on battery discharge only~\cite{song_lithium-ion_2018,mamo_long_2020,jiao_gru-rnn_2020,javid_adaptive_2020}, capturing complete charge-discharge cycle~\cite{Chemali2017}, multiple combinations at various temperatures or profiles~\cite{xiao_accurate_2019,mamo_long_2020,Chemali2017,javid_adaptive_2020}, the impact of data samples ammount~\cite{song_lithium-ion_2018} and cross-validation of all three current profiles against each other~\cite{mamo_long_2020}.
%
% Mamo et al.~\cite{mamo_long_2020} conducted experiments by validating the RNN model performance by training one driving profile and testing against the other two.
% The results showed doubled higher Root Mean Squared Error, as opposed to training and validation over single driving profiles as per experiments by Xiao et al.~\cite{xiao_accurate_2019}.
%
% \textcolor{red}{Follow the enumaeration}
% \begin{enumerate}
%     \item only discharge~\cite{song_lithium-ion_2018,mamo_long_2020,jiao_gru-rnn_2020,javid_adaptive_2020}
%     \item charge-discharge~\cite{Chemali2017}
%     \item multiple of cycles, like temps or profiles~\cite{xiao_accurate_2019,mamo_long_2020,Chemali2017,javid_adaptive_2020}
%     \item Limits of one cycle, benefits of multiple or length of it,  examples to compare to~\cite{mamo_long_2020,song_lithium-ion_2018}
% \end{enumerate}
Identifying the best-suited method for a specific condition, such as driving an EV, is a crucial step in machine learning engineering.
It requires a carefully defined methodology, which characterises the research conditions as closely as possible, and experimental results from multiple models with applicable techniques and the fewest errors.
By comparing the implementation and results from different sources and comparing testing accuracy and performance against multiple driving conditions at various temperatures, ranging from ambient to maximum possible, it is possible to select the best machine-learning technique which can be directly integrated  into an electric vehicle and safely used on both tight city roads and long, high-speed highways.
%There have not been many similar experiments against other SoC estimation methods since such validation procedures are necessary for the computer intelligence method compared to battery modelling.
% In addition, all investigations did poor research in extrapolating ideal laboratory conditions with battery cycler, stable temperatures and no sensor discrepancies to an actual driving situation on the road, mainly due to lack of car data.
%
% Usage of a single profile's battery cycling data may not validate ML methods' efficiency in driving an electric vehicle.
% The inability to determine the charge's current state during EV driving makes the online learning process inapplicable.
% Even with other SoC estimation technique usage, the computational complexity of training any NN is complicated to fit on a low-power device.
% An offline-trained model had the advantage of insignificant resource consumption during the prediction stage, making it a prefered way for an EV.
% All further model testing will be applied through varying battery cycling profiles to capture the influence of data type to model efficiency.
%
% (No point)
% The drawback lies within the validation procedures of produced models.
% Ideally, this process must be performed with similar data but under different conditions or at different times.
% However, once the produced model is taken through different unseen scenarios, the accuracy lowers drastically compared to early reported results since it starts to experience something completely unexpected.
% The lack of training samples or means to produce at different conditions affected the final judgment results of many publications. 
% Even though the estimation produces accurate output with tabled data, placed under actual driving conditions - the model will have difficulties matching training performance. 


%
%
% Mamo et al. conducted experiments
%
% ------------------------------------------------------------------------ \\
% PAPER CONTRIBUTION may be worth adding as a clear statement \\
% ------------------------------------------------------------------------ \\
% Asses handpicked articles from different categories of SoC estimation, asses efficiency and complexity to utilise in EV.
% This paper's contribution lies in researching recent approaches used to estimate Charge's State from battery sensor readings.
% Comparing and implementing the most promising algorithms intended to determine the direction to build upon in the future. \\ \\
% This paper will test recent developments in the most common SoC prediction to determine the most promising direction/approach towards implementing an improved light model applied to an electric vehicle.
% It determined to find the ML model creation's weakness and implement a new one for an EV. \\
% ------------------------------------------------------------------------ \\
%
%
% This paper investigates, implements, and compares extended memory-based models of RNN to predict the State of Charge and additional built-on over-time techniques to select the most effective and least resource-requiring onboard-based computations applicable within Electrical Vehicles.
\ifthenelse{\boolean{thesis}}{
    This chapter investigates, implements, and compares extended memory-based models of RNN to predict the State of Charge and additional built-on over-time techniques to select the most suitable practical application for EV use with combinations of different profiles.
} {
    This paper investigates, implements, and compares extended memory-based RNN models to predict the state of charge and additional built-on over-time techniques and select the most suitable, practical application for EV use with combinations of different profiles.
}
Each subset contains an implementation derived from various key references, changing either the structure of the models or the learning approach.
This should help us to develop a methodology that can further extrapolate offline trained methods from the lab condition to road-drive tests.
These algorithms have been used on all kinds of rechargeable batteries, but this research focuses on only one type of cell, which is openly available for everyone to access.
% The recent advancement and commonly used subsets are Gradient Recurrent Unit and Long Shot-Term Memory unit cells.
% Recurrent Neural Network has been confirmed suitable for the battery-related system by authors discussed further, such as Chemali~\cite{LSTM_Hochreiter1997}.
% However, there is no valid proof of if GRU or LSTM is helpful for battery SoC estimation.
% The best way to separate them and explore differences and efficiencies is to use them as Stateful and Stateless models.
The A123 lithium-ion battery data with three typical driving profiles, obtained from the University of Maryland 2012~\cite{noauthor_calce_2017} cycling experiment, acted as training and testing samples.
Several spreadsheets of the sensory measured values of an experimentally cycled single Li-ion cell have been obtained by the Universities' Battery Research Group using a battery cycler.
Each method is validated through these samples (either DST, US06, or FUDS driving profiles) and tested compared to the robustness and accuracy of state-of-charge estimations for batteries in the other two unseen schedulers.

%
%
\ifthenelse{\boolean{thesis}}{
    Since there has been no comparison between which RNN type or driving profile impacts State of Charge estimation for both charge and discharge cycles, this research aims to identify the most viable and optimum method for custom-build Electric Vehicles.
} {
    Since there has been no comparison to determine which RNN type or driving profile impacts the state-of-charge estimation for both charge and discharge cycles, this article aims to identify the most viable and optimum method for custom-built electric vehicles.
}
% Sometimes, multiple published models presented results relying only on discharge cycles, describing the efficiency of modelling on a single battery or in an ideal lab environment~\cite{xia_state_2018,javid_adaptive_2020,jiao_gru-rnn_2020,mamo_long_2020,zhang_deep_2020}.
However, long overnight charge cycles and regenerative breaking burst charges are equally crucial for the SoC percentage in the context of electric vehicles' battery utilisation with prolonged usage and influence the models' weight and biases.
The contribution of this work is in the implementation of all those methods in a comparable way, evaluated against real-world drive cycle scenarios using full charge--discharge cycles over various temperatures and extrapolated into different driving profiles.
It works out which is best for machine-learning SoC prediction and shows novelty in the testing validation procedures.
There have been many publications, but none tested these methods against each other under the same set of testing circumstances.
To move forward with either an effective SoC estimation or the development of a new SoC estimation algorithm,  the most effective algorithm for real-world scenarios must be determined.
This paper provides that evaluation and concludes which approach is the most appropriate.
% (2)This article will present a novel methodology for comparing multiple models with different techniques to each other, accommodating MLs' stochastic nature.
% (3)This article will compare multiple works in SoC estimation to identify the most applicable to car driving and prose methods to build on top further.
% }}
% ~\cite{Chemali2017,}
%
%
The remaining sections are organised as follows.
\ifthenelse{\boolean{thesis}}{
    Details on algorithms and optimisers are written in Section~\ref{sec:AN:Body}, where Subsection~\ref{subsec:structure} separates all details for each GRU and LSTM method, and Subsection~\ref{subsec:optimisers} breaks down every applied optimising algorithm.
    Applied methodology with details regarding training procedure and hyper-parameters selection are outlined in Section~\ref{sec:Meth}, with processing data in Subsection~\ref{subsec:b_data}.
    Section~\ref{sec:AN:Results} gives the results of implementation and performance characteristics and concludes the critical analysis, while Section~\ref{sec:AN:Conclussion} concludes the Chapter.
} {
    Details on algorithms and optimisers are written in Section~\ref{sec:Body}, where Section~\ref{subsec:structure} separates all details for each GRU and LSTM method, and Section~\ref{subsec:optimisers} breaks down every applied optimising algorithm.
    Applied methodology with details regarding training procedure and the selection of hyperparameters are outlined in Section~\ref{sec:Meth}, with processing data provided in  Section~\ref{subsec:b_data}.
    Section~\ref{sec:results} gives the results of the implementation and performance characteristics and concludes the critical analysis, while
    Section~\ref{sec:conclussion} concludes the article.
}