\subsection{Model training and evaluation} \label{subsec:training}
Evaluation process and etc.


% One of the simplest methods used to optimise the model is the Stochastic Gradient Descent (SGD), \mbox{Algorithm~\ref{alg:training}}.
% \begin{algorithm}\captionsetup{labelfont={sc,bf}, labelsep=newline}
%     \caption{Training procedure}
%         \begin{algorithmic}[1]
%             % \STATE \textbf{Input:} Data sample with shape=(1,500,1,3) 
%             % \STATE \textbf{Output:} Predicted SoC shape=(1,1,1)
%             \STATE Setup model. Define optimiser and metrics.
%             \STATE Initialise parameters with initial learning rate at 0.001
%             \STATE Set prev\_error 1
%             \WHILE{epoch  $<$100:}
%                 \STATE Train model, get gradients and apply optimiser
%                 \IF{ error $<$ prev\_error:}
%                     \WHILE{attempt  $<$ 50:}
%                         \STATE Load previos succesful model
%                         \STATE Reduce learning rate by half
%                         \STATE Train model, get gradients and apply optimiser
%                         \IF{ error $>$ prev\_error:}
%                             \STATE Break the loop. Update error. Save state.
%                         \ENDIF
%                     \ENDWHILE
%                 \ELSE
%                     \STATE Update error. Save state.
%                     \STATE Update learning rate based on sheduler
%                 \ENDIF
%                 \STATE Validate model on 25 $degrees$ cycler.
%                 \STATE Test on two other profiles.
%                 \STATE Test on different cell.
%             \ENDWHILE
%             \STATE Record overall results againt entire training datasets
%         \end{algorithmic}
%     \label{alg:training}
% \end{algorithm}