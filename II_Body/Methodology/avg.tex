\subsection{Results averaging} \label{subsec:avg}
% Stochastic nature of ML and Averaging 
%
%* Most deep learning algorithms are based on an optimization algorithm called stochastic gradient descent. — Page 98, Deep Learning, 2016.
The five models of this research have a stochastic nature due to the use of randomness in the calculation of the gradient during the learning~\cite{alma991010036879604001}, like Stochastic Gradient Descent, which is discussed in Subsection~\ref{subsec:optimisers}.
As such, getting repeatable results, worthy of comparison between each other, requires the implementation of an averaging method, where each model will undertake the same procedures multiple times.
The resulting plots and values provide far more representative statistical criteria, as opposed to random fluctuations.
% \textcolor{red}{5 ML is stochastict thongs. As such, to get repeatable you need to account for statistics, to account, we have implemented averaging methods, where each model is trained 10 times and results averaged to provide significant statistical criteria (representative) as opposed to random fluctuations/
%As an example of this method FUDS has been used, showing a spread....}

%
% Average 10 attempts with accuracy calculus.
During preliminary work, the research attempted to train many models of the same type to produce a best-fitting line.
%Due to MLs' stochastic nature,
The training results had a huge variance, similar to Subfigure~\ref{subfig:Model-DEMO-10} and ~\ref{subfig:Model-DEMO2-10}.
The Earliest attempt to unify all models was the application of the same set seed number for all models.
However, determining a case from an infinite number of possibilities to produce equally effective models is the least effective method.
Therefore, an average result of multiple attempts to smoothen history and SoC prediction curves has been introduced.
Figures~\ref{fig:Model-DEMO} and~\ref{fig:Model-DEMO2} represent the difference between single and multiple averaged models and the general spread.
Subfigure~\ref{subfig:Model-DEMO-10} demonstrates how ten different models, each potentially minimised toward a single minimal, get spread from each other, producing variable results.
Similar case with SoC prediction on Subfigure~\ref{subfig:Model-DEMO2-10}, where combined predictions and RMS highlight potential inefficancies in the prediction of the same task 
Subfigures~\ref{subfig:Model-DEMO-1} and~\ref{subfig:Model-DEMO-avr} show a comparison between randomly selected single and averaged model history results.
Whereas Subfigures~\ref{subfig:Model-DEMO2-1} and~\ref{subfig:Model-DEMO2-avr} outline that the best case does bring subjective results to the efficiency of the techniques in terms of accurate results.
Due to the most trendiness plots and apparent convergence or divergence inclination, the method has proven to be the most objective result.
\textit{NaN} values were ignored in the cases of prolonged history.
The same method has been applied to all SoC plots for all models, reported further.
%Several evaluation possibilities were tested. In the end, an average of 10 provided the most objective results to overview.
%\textcolor{red}{Show plot of 1, then all 10 and average took.}
\begin{figure*}[htbp]
  \centering
  \begin{subfigure}[b]{0.325\textwidth}
      \centering
      \includesvg[width=\linewidth]{III_Conclussion/Results_new/DEMO1-history-FUDS-mae.svg}
      \caption{Single model history for training and testing}
      \label{subfig:Model-DEMO-1}
  \end{subfigure}
  \hfill
  \begin{subfigure}[b]{0.325\textwidth}
      \centering
      \includesvg[width=\linewidth]{III_Conclussion/Results_new/DEMO2-history-FUDS-mae.svg}
      \caption{All 10 attempts of histories on a single plot for training and testing}
      \label{subfig:Model-DEMO-10}
  \end{subfigure}
  \hfill
  \begin{subfigure}[b]{0.325\textwidth}
      \centering
      \includesvg[width=\linewidth]{III_Conclussion/Results_new/DEMO3-history-FUDS-mae.svg}
      \caption{Average of 10 attempts of histories for training and testing}
      \label{subfig:Model-DEMO-avr}
  \end{subfigure}
  \caption{History results averaging demonstration}
  \label{fig:Model-DEMO}
\end{figure*}
\begin{figure*}[htbp]
  \centering
  \begin{subfigure}[b]{0.325\textwidth}
      \centering
      \includesvg[width=\linewidth]{III_Conclussion/Results_new/DEMO1-SoC-FUDS.svg}
      \caption{Single model SoC prediction for training \ \ \ \ \ }
      \label{subfig:Model-DEMO2-1}
  \end{subfigure}
  \hfill
  \begin{subfigure}[b]{0.325\textwidth}
      \centering
      \includesvg[width=\linewidth]{III_Conclussion/Results_new/DEMO2-SoC-FUDS.svg}
      \caption{All 10 attempts of SoC prediction on a single plot for training}
      \label{subfig:Model-DEMO2-10}
  \end{subfigure}
  \hfill
  \begin{subfigure}[b]{0.325\textwidth}
      \centering
      \includesvg[width=\linewidth]{III_Conclussion/Results_new/DEMO3-SoC-FUDS.svg}
      \caption{Average of 10 attempts of SoC prediction for training}
      \label{subfig:Model-DEMO2-avr}
  \end{subfigure}
  \caption{State of Charge results averaging demonstration}
  \label{fig:Model-DEMO2}
\end{figure*}