\subsection{Results averaging} \label{subsec:avg}
% Stochastic nature of ML and Averaging 
%
%* Most deep learning algorithms are based on an optimization algorithm called stochastic gradient descent. — Page 98, Deep Learning, 2016.
The five models used in this research had a stochastic nature due to the use of randomness in the  gradient calculation during the learning~\cite{alma991010036879604001}, such as the stochastic gradient descent, which is discussed in Section~\ref{subsec:optimisers}.
As such, obtaining repeatable results which were worth comparing required the implementation of an averaging method, in which each model undertook the same procedures multiple times.
The resulting plots and values provided far more representative statistical criteria, as opposed to random fluctuations.
% \textcolor{red}{5 ML is stochastict thongs. As such, to get repeatable you need to account for statistics, to account, we have implemented averaging methods, where each model is trained 10 times and results averaged to provide significant statistical criteria (representative) as opposed to random fluctuations/
%As an example of this method FUDS has been used, showing a spread....}

% Average 10 attempts with accuracy calculus.
%
During preliminary work, this research attempted to train many models of the same type to produce the best-fitting line.
%Due to MLs' stochastic nature,
The training results showed a significant variance.
As such, training was repeated ten times for each dataset to remove the statistical variance from the comparison results, and the average of all ten trained models was used in each result.
Figures~\ref{fig:Model-DEMO} and~\ref{fig:Model-DEMO2} show this process, where Figures \ref{fig:Model-DEMO}a and \ref{fig:Model-DEMO2}a show a single test training history and final SoC prediction, Figures \ref{fig:Model-DEMO}b and \ref{fig:Model-DEMO2}b show the spread of 10 similar training sessions, and Figures \ref{fig:Model-DEMO}c and \ref{fig:Model-DEMO2}c show the averaged result.
It is apparent that, while some repeats were highly accurate, others had more significant errors, see Figure~\ref{fig:Model-DEMO}b.
The averaging provided a representative, statistically repeatable result.
% similar to Subfigure~\ref{subfig:Model-DEMO-10} and ~\ref{subfig:Model-DEMO2-10}.
% The Earliest attempt to unify all models was the application of the same set seed number for all models.
% However, determining a case from an infinite number of possibilities to produce equally effective models is the least effective method.
% Therefore, an average result of multiple attempts to smoothen history and SoC prediction curves has been introduced.
% Figures~\ref{fig:Model-DEMO} and~\ref{fig:Model-DEMO2} represent the difference between single and multiple averaged models and the general spread.
% Subfigure~\ref{subfig:Model-DEMO-10} demonstrates how ten different models, each potentially minimised toward a single minimal, get spread from each other, producing variable results.
% Similar case with SoC prediction on Subfigure~\ref{subfig:Model-DEMO2-10}, where combined predictions and RMS highlight potential inefficancies in the prediction of the same task 
% Subfigures~\ref{subfig:Model-DEMO-1} and~\ref{subfig:Model-DEMO-avr} show a comparison between randomly selected single and averaged model history results.
% Whereas Subfigures~\ref{subfig:Model-DEMO2-1} and~\ref{subfig:Model-DEMO2-avr} outline that the best case does bring subjective results to the efficiency of the techniques in terms of accurate results.
% Due to the most trendiness plots and apparent convergence or divergence inclination, the method has proven to be the most objective result.
% \textit{NaN} values were ignored in the cases of prolonged history.
% The same method has been applied to all SoC plots for all models, reported further.
%Several evaluation possibilities were tested. In the end, an average of 10 provided the most objective results to overview.
%\textcolor{red}{Show plot of 1, then all 10 and average took.}
\begin{figure}[H]
    % \centering
    \begin{subfigure}[b]{0.325\textwidth}
        % \centering
        \includesvg[width=4.5cm]{III_Conclussion/Results_new/DEMO1-history-FUDS-mae.svg}
        \caption{\centering}
        \label{subfig:Model-DEMO-1}
    \end{subfigure}
    % \hfill
    \begin{subfigure}[b]{0.325\textwidth}
        % \centering
        \includesvg[width=4.5cm]{III_Conclussion/Results_new/DEMO2-history-FUDS-mae.svg}
        \caption{\centering}
        \label{subfig:Model-DEMO-10}
    \end{subfigure}
    % \hfill
    \begin{subfigure}[b]{0.325\textwidth}
        % \centering
        \includesvg[width=4.5cm]{III_Conclussion/Results_new/DEMO3-history-FUDS-mae.svg}
        \caption{\centering}
        \label{subfig:Model-DEMO-avr}
    \end{subfigure}
    \caption{History results averaging demonstration. (\textbf{a}) Single model history for training and testing; (\textbf{b}) All 10 attempts of histories for training and testing; (\textbf{c}) Average of 10 attempts of histories for training and testing.}
    \label{fig:Model-DEMO}
\end{figure}
\begin{figure}[H] %Attention AE: please change SoC(%) to SoC (%).
    % \centering
    \begin{subfigure}[b]{0.325\textwidth}
        % \centering
        \includesvg[width=4.5cm]{III_Conclussion/Results_new/DEMO1-SoC-FUDS.svg}
        \caption{\centering}
        \label{subfig:Model-DEMO2-1}
    \end{subfigure}
    %\hfill
    \begin{subfigure}[b]{0.325\textwidth}
        % \centering
        \includesvg[width=4.5cm]{III_Conclussion/Results_new/DEMO2-SoC-FUDS.svg}
        \caption{\centering}
        \label{subfig:Model-DEMO2-10}
    \end{subfigure}
    % \hfill
    \begin{subfigure}[b]{0.325\textwidth}
        % \centering
        \includesvg[width=4.5cm]{III_Conclussion/Results_new/DEMO3-SoC-FUDS.svg}
        \caption{\centering}
        \label{subfig:Model-DEMO2-avr}
    \end{subfigure}
    \caption{State of charge results' averaging demonstration. (\textbf{a}) Single-model SoC prediction for training; (\textbf{b}) all 10 attempts a SoC prediction on a single plot; (\textbf{c}) average of 10 attempts at SoC prediction for training.}
    \label{fig:Model-DEMO2}
\end{figure}

%
%
\ifthenelse{\boolean{thesis}}{
    In the next Section~\ref{sec:AN:Results}, first tests to determine optimum hyperparameters are discussed; then, the full evaluation of models against each of the drive cycles is presented.
} {
    In the next Section~\ref{sec:results}, the first tests carried out to determine the optimum hyperparameters are discussed; then, the full evaluation of models against each of the drive cycles is presented.
    In each case, the train/validation/test procedures and error metrics of \mbox{Section~\ref{subsec:t_model}} were followed, with the learning rate method in Section~\ref{subsec:l-rate} and an average of 10 training sessions as in Section~\ref{subsec:avg} to ensure the best and most accurate representation of each model.
}