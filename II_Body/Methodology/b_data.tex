\subsection{Battery Data for training and validation} \label{subsec:b_data}
% Battery data description and overview. **Table III**
%
Model training is to be conducted over Lithium-Ion battery cycling data obtained by the Battery Research Group of the Center for Advanced Life Cycle Engineering (CALCE) Group at the University of Maryland~\cite{noauthor_calce_2017} in 2012.
According to the associated paper, the battery cycling has been performed with an Arbin BT2000 tester machine and controlled with official Arbin Mits Pro Software (v4.27)~\cite{xing_state_2014}.
% \textcolor{red}{\textbf{The value of charge has not matched with battery testing of the similar cells conducted during this research with freshly bought M1-B series cells.
% Despite their small difference, the CALCE most likely used a series of two cells with either doubled current or 1000 cycles used batteries before publishing the data with reduced capacity.}}
\mbox{Table~\ref{tab:battery}} highlights selected battery characteristics directly from datasheet~\cite{noauthor_anr26650m1a}.
\ifthenelse{\boolean{thesis}}{
\begin{table}[ht]
    \renewcommand{\arraystretch}{1.3}
    \caption{Battery characteristics}
    \centering
    \label{tab:battery}
    \begin{tabular}{ l c c c c c }
        \hline\hline \\[-3mm]
        Brand & Cell      & Cell & Battery & Nominal          & Charge/discharge\\
        name  & Chemistry & Type & Weight  & Capacity $C_{N}$ & cut-off voltage \\
        % \makecell{Brand name} & \makecell{Cell Chemistry} & \makecell{Cell Type} & \makecell{Battery Weight} & \makecell{Nominal Capacity $C_{N}$} & \makecell{Nominal Voltage} & \makecell{Charge/discharge cut-off voltage} \\
        \hline
        %A456 \\ (former A123) & 76g+-1g & 2.3Ah & 3.2V & 3.65V, 2.0 V\\
        A123   & LiFePO4 & ANR26650 & 70g        & 2.3Ah & 3.6V\\
        (2012) &         & \textbf{M1-A}     & \textpm 2g &       & 2.0V \\
        \hline\hline
    \end{tabular}
\end{table}
} {
\begin{table}[H]
    \caption{Battery characteristics}
    \centering
    \label{tab:battery}
    \newcolumntype{C}{>{\centering\arraybackslash}X}
    \begin{tabularx}{\textwidth}{ C C C C C C }
        \toprule
        \textbf{Brand} & \textbf{Cell} & \textbf{Cell} & \textbf{Battery} & \textbf{Nominal} & \textbf{Charge/discharge} \\
        \textbf{name}  & \textbf{Chemistry} & \textbf{Type} & \textbf{Weight}  & \textbf{Capacity $C_{N}$} & \textbf{cut-off voltage} \\
        % \makecell{Brand name} & \makecell{Cell Chemistry} & \makecell{Cell Type} & \makecell{Battery Weight} & \makecell{Nominal Capacity $C_{N}$} & \makecell{Nominal Voltage} & \makecell{Charge/discharge cut-off voltage} \\
        \midrule
        %A456 \\ (former A123) & 76g+-1g & 2.3Ah & 3.2V & 3.65V, 2.0 V\\
        A123   & LiFePO4 & ANR26650 & 70g        & 2.3Ah & 3.6V\\
        (2012) &         & M1-A     & \textpm 2g &       & 2.0V \\
        \bottomrule
    \end{tabularx}
\end{table}
}
%
% How to work with battery data and which were selected.
The Battery Cycling data over 2 Li-Ion cells were stored as Excel spreadsheets over temperature range 0\textdegree{}C and 50\textdegree{}C degrees, with 10 degree steps and tolerance of around 0.5-1\textdegree{}C, including an ambient temperature of 25\textdegree{}C.
Each testing cycle contains three profiles, distinguished by their current consumption, emulating a stress test or driving scenarios: Dynamic stress test (DST) - Figure~\ref{subfig:profs-DST}, Highway (US06) - Figure~\ref{subfig:profs-US} and the Federal urban driving schedules (FUDS) - Figure~\ref{subfig:profs-FUDS}.
Each cycle consists of charge and discharge periods, with a sampling rate of 4Hz and 1Hz, respectively.
Charging periods have been linearly interpolated to match the data sampling rate.
%without additional random noise due to insignificant impact.
The range of 20\textdegree{}C to 50\textdegree{}C was used as a training and validation dataset since, this is the most common temperature range for EVs involved in the research.
It results in $\sim$58,613 and $\sim$12,171 samples for training and validation over one profile.
Each model from \mbox{Table~\ref{tab:experiment}} was trained independantly on each drive cycle profile and then tested against the other two, as per Mamo and Wang~\cite{mamo_long_2020}.
The performance calculation has been conducted over two cycles of 30\textdegree{}C and 40\textdegree{}C samples for each of the two remaining profiles, leading to a total of 47,022 testing samples.
% After compleating three models per profile for every evaluated method, the overall performance calculation has been conducted against all samples of all three profiles, making a good performance capture for any possible driving conditions.
\begin{figure*}[htbp]
    \centering
    \begin{subfigure}[b]{0.28\textheight}
        \centering
        \includesvg[width=\linewidth]{II_Body/images/Current-DST.svg}
        \caption{Dynamic Stress Test (DST)}
        \label{subfig:profs-DST}
    \end{subfigure}
    \hfill
    \begin{subfigure}[b]{0.28\textheight}
        \centering
        \includesvg[width=\linewidth]{II_Body/images/Current-US06.svg}
        \caption{Highway driving schedule (US06)}
        \label{subfig:profs-US}
    \end{subfigure}
    \hfill
    \begin{subfigure}[b]{0.28\textheight}
        \centering
        \includesvg[width=\linewidth]{II_Body/images/Current-FUDS.svg}
        \caption{Federal urban driving schedule (FUDS)}
        \label{subfig:profs-FUDS}
    \end{subfigure}
    \caption{Cell Current of three battery testing profiles, emulating Constant-Current-Constant-Voltage charge and regenerative discharge until cells reached top or bottom cut-offs.}
    \label{fig:current-profs}
\end{figure*}

% State of charge calculation with equations.
%
Like in any battery usage scenario, the State of Charge has not been provided in the CALCE data.
However, the Arbin machine stores both in and out charges as separate arrays, along with the applied current.
The SoC value can be calculated from the difference between Charge and Discharge Capacities in $Ah$.
% The \textit{MinMax} algorithm allows adjustment of the minimum and maximum point of the SoC to be within bounds 0 and 100\%.
% \begin{equation}
%     \begin{split}
%         \hat{SoC} &= MinMax(C-D)
%         \label{eq:SoC-DC}
%     \end{split}
% \end{equation}
The resulting trend can be validated with Coulomb Counting, using the integral of the consumed and/or produced current $I$ between initial $t_0$ and the end of cycle $t_n$ time, divided by (converted to seconds) Nominal Capacity $C_{N}$ of the batteries, as per \mbox{Equation~\ref{eq:SoC-calc}}.
\begin{equation}
    \begin{split}
        % \hat{SoC} &= \frac{\int_{t_0}^{t_n} I(t)dt} {C_{N}} = \frac{\int_{t_0}^{t_n} I(t)dt} {2.3*3600} \\
        \hat{SoC} &= \frac{\int_{t_0}^{t_n} I(t)dt} {C_{N}} = \frac{\int_{t_0}^{t_n} I(t)dt} {(2.3*3600)} \\
        \label{eq:SoC-calc}
    \end{split}
\end{equation}
The final expected value gets rounded to two decimal places in all scenarios to simplify training and testing processes.
% , as per Equation~\ref{eq:SoC-round}.
% \begin{equation}
%     \begin{split}
%         SoC &= \frac{int(100\times\hat{SoC})}{100}
%         \label{eq:SoC-round}
%     \end{split}
% \end{equation}
%! Elaborate that at different temperatures, which produces above 100%
% Optionally, all calculated SoC can be adjusted using \textit{MinMax} scaler algorithm to set bounds between 0-100\%, \mbox{Equation~\ref{eq:MinMax}}.
% \begin{equation}
%     \begin{split}
%         SoC_{0\%} + \frac{ \left(SoC - \min \left( SoC \right)\right) \left( SoC_{100\%} - SoC_{0\%} \right) }
%                          { \max \left( SoC \right) - \min \left( SoC \right)} \\
%     \end{split}
%     \label{eq:MinMax}
% \end{equation}

%%%% The methodology unification with other authors will not be used in the current research since the quality and age of the tested batteries cannot be verified.
