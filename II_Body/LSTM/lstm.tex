%  Long-Short Term Memory based
% Definition of the LSTM
%
\subsubsection{Long-Short Term Memory (LSTM) based models} \label{subsub:lstm}
The most commonly used in the Time-series Machine learning modelling is the Long Short-Term Memory Cell~\cite{LSTM_Hochreiter1997}.
Like GRU, LSTM models preserve long-term dependencies in the extended data sequences.
For its' ten years of existence, it became the most widely used type of RNN in those applications.
\mbox{Figure~\ref{fig:LSTM-cell}} summarises the internal cell logic.
%Currently, the most common usage of the Time-series Machine Learning model is the prediction of stock prices, weather prognostic or any other time-dependent data.
%However, vanishing gradient is the most common problem for any of those scenarios.
% Long-range data tend to fade away from the model, which impacts overall prediction.
\begin{figure}[ht]%[htbp]
    \centering
    \includegraphics[width=\linewidth]{II_Body/LSTM/images/LSTM.jpg}
    \caption{Structure of Long Short-Term Memory unit cell~\cite{Hochreiter:1997:LSM:1246443.1246450}.}
    \label{fig:LSTM-cell}
\end{figure}
Unlike GRU, this cell utilises three gates instead of 2.
The update gate gets replaced with separate input $i_t$ and output $o_t$, as per \mbox{Equation~(\ref{eq:LSTM-gates})}.
All gets utilise the same sigmoid \mbox{Equation~(\ref{eq:sigmoid})}.
%\textcolor{red}{Those are classical LSTM approaches. Using history sizes, no Stateless methods.}
%Unlike with GRU, the update gates are renamed as an input gate $i_t$, with the same equation. Even that the forget gate $f_t$ is the same, model utilises another one, output gate $g_t$ as per Equations~\ref{eq:LSTM-gates}.
\begin{equation}
    \begin{split}
        f_t &= \sigma \left(W_f \left[h_{t-1}, x_t \right] + b_f \right) \\
        i_t &= \sigma \left(W_i \left[h_{t-1}, x_t \right] + b_i \right) \\
        o_t &= \sigma \left(W_o \left[h_{t-1}, x_t \right] + b_o \right) \\    
    \end{split}
    \label{eq:LSTM-gates}
\end{equation}
The main difference between LSTM and GRU lies in the cell state calculation.
Using the same $tanh$ activation function, \mbox{Equation~(\ref{eq:LSTM-output})} describes how cells will be updated and propagated further.
The $c_t$ represents the cell state at a timestamp.
\begin{equation}
    \begin{split}
        % \hat{c_t} &= tanh \left(W_c \left[h_{t-1}, x_t \right] + b_c \right) \\
        % c_t &= f_t c_{t-1}+i_t \hat{c_t} \\
        c_t &= f_t c_{t-1}+i_t \times tanh \left(W_c \left[h_{t-1}, x_t \right] + b_c \right) \\
        h_t &= o_t*tanh \left(c_t \right)
    \end{split}
    \label{eq:LSTM-output}
\end{equation}
% $c_t \rightarrow$ cell state (memory) at timestep $t$ \\
% $\hat{c_t} \rightarrow$ candidate for cell state \\
% $* \rightarrow$ element-wise multiplication \\
% Like the GRU cell type, the model training Library supports stateful and stateless utilisation of the LSTM model.
% It will be used as a Stateless cell for further comparison based on Chemali et al.~\cite{Chemali2017} and similar articles.

%  Attention layer
% Attention intro and explanation
% (7) \textcolor{red}{The most complicated one}. The method by WeiZhang2020. Adaptive Time-series prediction on online validation. Data were taken directly during cycling batteries.
% \textbf{I have to study this properly first. Long-Horizon, as they called it, was more useful to the State of Health. This should be the end of them.}

%  Attention Layer
% Introduction and origins of attention usage
%
\subsubsection{LSTM with Attention Layer}
The research conducted by Mamo \textit{et al.}~\cite{mamo_long_2020} was intended to determine weaknesses and improve LSTM structure by introducing additional techniques into the default layer of the training model. 
They added an Attention Layer~\cite{yang_hierarchical_2016} between LSTM and fully connected layers to improve accuracy and replace traditional gradient optimiser with probability-based Differential Evolution.
\mbox{Figure~\ref{fig:attention}} summarises the model structure, and \mbox{Equations~(\ref{eq:AttentionWithContext})} and~(\ref{eq:Addition}) define the internal logic between hidden layers and output.

% The origins of the source code
%
%bla bla bla... I am exosted, I hate this part; I have no idea what to write and have no desire to research more.
The implementation of the Attention layer has not been provided with the Machine Learning library at this point.
The source code from Winata research~\cite{winata_attention-based_2018} has been used instead.
The open-source code is publicly accessible through Github source~\cite{attention_8461990}.
Details in terms of optimiser usage and replacement are justified in subsection~\ref{subsec:optimisers}.
In the State of Charge estimation, the attention layer addresses two shortcomings of LSTM: replacing the traditional method of recursively constructing LSTM depth and located after the output of the primary layer, just before the model Dense layer output~\cite{mamo_long_2020}.
\begin{figure}[htbp]
    \centering
    \includegraphics[width=0.35\linewidth]{II_Body/LSTM/images/AttenrionDrawing.jpg}
    \caption{Attention based architecture}
    \label{fig:attention}
\end{figure}
\begin{equation}
    \begin{split}
        \hat{u_t} &= tanh \left(W h_{t} + b \right) \\
             \alpha_t &= \frac{exp(u^T u)}{\sum_t(exp(u_t^T u))} \\
              v_t &= \alpha_t*h_t%,\ v\ in\ time\ t
    \end{split}
    \label{eq:AttentionWithContext}
\end{equation}
\begin{equation}
    \begin{split}
        v = \sum_t(\alpha_t * h_t)
    \end{split}
    \label{eq:Addition}
\end{equation}
%
% \subsection{Implementation}
%     Following table highlights parameters which provided the best results for their experiments.
%     \textbf{Table of the parameters.}
%     The network will be the one discussed above.
%     Model itself will be multi-feature based with follwoing parameters: Voltage \textit{V(t)}, Current \textit{A(t)} and Temperature \textit{C(t)}, where \textit{t} represents a time-stamp. Each feature will contain an equal amount of sample and feed in the input column vector. As a result, a single input will have the following form: \\
%     % [V(0)], [I(0)], [T(0)] \\
%     % [V(1)], [I(1)], [T(1)] \\
%     % [V(n)], [I(n)], [T(n)] \\
%     where \textit{n} is the history size.
%     The output vector will be a State of Charge \textit{SoC(\%)} percentage up to 2 decimal places, within range 0 to 1 and time stamp \textit{n}: \\
%     % [SoC(n)] \\
%     As a result, the shape of input and output data will be: X(0)=(n,3) and Y(0)=(1)
%     The entire dataset will use a single-step windowing technique with no batches to save memory and utilise \textbf{stateless}* \footnote{Need to discuss this with Holmes.} model. \\
%     The Generated dataset of sample size \textit{k}, will consist of two Tensor input/output Vectors of the following shape: \\
%     X = (k-n,n,3), Y = (k-n,1).
%     The variable type for computation was selected to be float32.
%     To keep the Denerated dataset simple, no batching approach spared from dealing with 4-dimensional input vectors.
% \subsection{Prediction results}


%     Implementation of the model based on Chemali2017—application for our section and results. Refer to methodology from time to time.