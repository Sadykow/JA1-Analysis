\section{Long-Short Term Memory (LSTM)}\label{sec:LSTM}
\subsection{Definition}
    Currently, the most common usage of the Time-series Machine LEarning model is the prediction of stock prices, weather prognosition or any other time dependant data.
    However, the most common problem for any of those scenarios is vanishing gradient.
    Long range data tend to fade away from the model, which impacts overall prediction.
    The Long Short-Term Memory (LSTM) models tend to capture long-term \textbf{dependancies by longer preserving the sequence} of data. \\
    The structure of LSTM Recurrent Neural Network is similar to clasical RNN, discussed earlier.
    The difference is withing cells themself.
    Unlike Dense layers with single activation function, the structure LSTM, presented on \textbf{Figure} consists of multiple gates.
    \textbf{Try put very brief desciption from my notebook on structure followed by formulas.} \\ [2 pc]
\subsection{Implementation}
    Following implementation is based on Chemali2017 article, which used similar setup and dataset to predict State of Charge with different Network depts \footnote{history sizes}.
    Following table higlighs parameters, which provided best results for their experements.
    \textbf{Table of the parameters.}
    The network will be the one discussed above.
    Model itself will be multi-feature based with follwoing parameters: Voltage \textit{V(t)}, Current \textit{A(t)} and Temperature \textit{C(t)}, where \textit{t} represents a time-stamp. Each feature will contain equal amount of sample and each will be feed in input column vector. As a result, a single input will have following form: \\
    % [V(0)], [I(0)], [T(0)] \\
    % [V(1)], [I(1)], [T(1)] \\
    % [V(n)], [I(n)], [T(n)] \\
    where \textit{n} is the history size.
    The output vector will be a State of Charge \textit{SoC(\%)} percentage up to 2 decimal places, within range 0 to 1 and time stamp \textit{n}: \\
    % [SoC(n)] \\
    As a result the shape of input and output data will be: X(0)=(n,3) and Y(0)=(1)
    The entire dataset will use single-step windowing tecnhinue with no bathces to save memory and utilsase \textbf{stateless}* \footnote{Need to discuss this with Holmes.} model. \\
    The Generated dataset of sample size \textit{k}, will consist of two Tensor input/output Vectors of following shape: \\
    X = (k-n,n,3), Y = (k-n,1).
    The variable type for computation was selected to be float32.
    To keep Denerated dataset simple, no batching approach spared from dealing with 4-dimensional input vectors.
\subsection{Prediction results}


    Implementation of the model based on Chemali2017. Application for our section and results. Refer to methodology from time to time.