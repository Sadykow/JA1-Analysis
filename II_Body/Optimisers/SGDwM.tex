\subsubsection{Classic Stochastic Gradient and Momentum Stochastic Gradient algorithm}
One of the simplest methods used to optimise the model is the Stochastic Gradient Descent (SGD), Algorithm~\ref{alg:SGD}.
\begin{algorithm}\captionsetup{labelfont={sc,bf}, labelsep=newline}
  \caption{Stochastic Gradient Descent (SGD) optimisation}
  \begin{algorithmic}[1]
    \STATE \textbf{Number of input samples} \\ $N\gets length(\textit{input data})$\\
    \STATE \textbf{Size of windows} \\ $S\gets length(V_{i..n})$\\
    \STATE Input: $x_n = [V_{i..n}, I_{i..n}, T_{i..n}] - $Shape: $X = (N, S, 3)$
    \STATE Output:$y_n = [SoC_{n}] - $Shape:$Y = (N, 1)$
    \STATE Define Loss function: $L$ \\
           Get hyperparameters: $\alpha$
    \WHILE{$W_t \text{ not converge}$}
    \STATE $t \gets t+1$
    \STATE $g_t \gets \nabla_\phi L_t (W_{t-1})$ \COMMENT{Obtain gradient}
    \STATE $W_t \gets W_{t-1} - \alpha g_t $ \COMMENT{Update parameters}
    \ENDWHILE
  \end{algorithmic}
  \label{alg:SGD}
\end{algorithm}
The SGD Optimiser utilised a simple gradient update with a learning rate.
Unlike improved versions, this algorithm has the potential of missing optimum value.
The extension of SGD, Algorithm~\ref{alg:SGDwM}, which Jiao et al.~\cite{jiao_gru-rnn_2020} used, applies to classical SGD a single momentum calculation, Line~\ref{alg:SGDwM-Line-Moment}.
\begin{algorithm}\captionsetup{labelfont={sc,bf}, labelsep=newline}
  \caption{Stochastic Gradient Descent with Momentum optimisation}
  \begin{algorithmic}[1]
    \STATE \textbf{Number of input samples} \\ $N\gets length(\textit{input data})$\\
    \STATE \textbf{Size of windows} \\ $S\gets length(V_{i..n})$\\
    \STATE Input: $x_n = [V_{i..n}, I_{i..n}, T_{i..n}] - $Shape: $X = (N, S, 3)$
    \STATE Output:$y_n = [SoC_{n}] - $Shape:$Y = (N, 1)$
    \STATE Define Loss function: $L$ \\
           Get hyperparameters: $\alpha, \beta_1$
    \WHILE{$W_t \text{ not converge}$}
    \STATE $t \gets t+1$
    \STATE $g_t \gets \nabla_\phi L_t (W_{t-1})$ \COMMENT{Obtain gradient}
    \STATE $m_t \gets \beta_1 m_{t-1}+(1-\beta_1) g_t $ \COMMENT{$1_{st}$ Moment Calculation\label{alg:SGDwM-Line-Moment}}
    \STATE $W_t \gets W_{t-1} - \alpha m_t $  \COMMENT{Update parameters}
    \ENDWHILE
  \end{algorithmic}
  \label{alg:SGDwM}
\end{algorithm}

%
%
To improve accuracy, Jiao introduced noise to the data to be able to capture more variant information.
Due to the amount of data provided and comparison with other methods, noise variance will not be used in this implementation.
Table~\ref{tab:params-jiao} provides parameters selection for the optimiser.
\begin{table}[ht]
    \renewcommand{\arraystretch}{1.3}
    \caption{Hyper-Parameters as per Jiao et al.~\cite{jiao_gru-rnn_2020}}
    \centering
    \label{tab:params-jiao}
    \resizebox{\columnwidth}{!}{
    \begin{tabular}{ l c c }
      \hline\hline \\[-3mm]
        Method     & $\alpha$ & $\beta_1 $  \\
        \hline
        SGDw/M
                & $0.001$ & $0.8$  \\% 0.0000001
        \hline\hline
    \end{tabular}
    }
\end{table}
\newpage