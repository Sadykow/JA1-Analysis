\subsubsection{Ensemble optimisation with Nadam and AdaMax}
The Adam algorithm remains the most commonly used optimiser.
The reason behind changing it to another lies within two potential issues.
The first is the training converges, which may not happen at all~\cite{reddi_convergence_2019}.
The second is the chance of missing the optimal solution~\cite{wilson_marginal_2017}.
Xioa et al.~\cite{xiao_accurate_2019} proposed a novel alternative of combining several optimisers to address those issues.
The new ensemble optimisation algorithm is based on the combination of Nadam~\cite{dozat_nadam_2016} and the AdaMax~\cite{kingma_adam_2017} at certain moments of training.

%
%
The Nadam optimiser extends the Adam, \mbox{Algorithm~\ref{alg:nadam}}, which implements the Nesterov momentum~\cite{dozat_nadam_2016}.
\begin{algorithm}\captionsetup{labelfont={sc,bf}, labelsep=newline}
    \caption{Nesterov Adaptive Moment Estimation (Nadam) optimisation}
    \begin{algorithmic}[1]
      \STATE \textbf{Number of input samples} \\ $N\gets length(\textit{input data})$\\
      \STATE \textbf{Size of windows} \\ $S\gets length(V_{i..n})$\\
      \STATE Input: $x_n = [V_{i..n}, I_{i..n}, T_{i..n}] - $Shape: $X = (N, S, 3)$
      \STATE Output:$y_n = [SoC_{n}] - $Shape:$Y = (N, 1)$
      \STATE Define Loss function: $L$ \\
             Get hyperparameters: $\alpha, \beta_1, \beta_2, \epsilon$
      \WHILE{$W_t \text{ not converge}$}
      \STATE $t \gets t+1$
      \STATE $g_t \gets \nabla_\phi L_t (W_{t-1})$ \COMMENT{Obtain gradient}
      \STATE $m_t \gets \beta_1 m_{t-1}+(1-\beta_1) g_t $ \COMMENT{$1_{st}$ moment calculation}
      \STATE $\upsilon_t \gets \beta_2 \upsilon_{t-1}+ \left(1-\beta_2 \right)g^2_t $ \COMMENT{$2_{nd}$ moment calculation}
      \STATE $\hat{m_t} \gets \frac{m_t}{1-\beta^t_1}$ \COMMENT{Corrected $\hat{m_t}$}
      \STATE $\hat{\upsilon_t} \gets \frac{\upsilon_t}{1-\beta^t_2} $ \COMMENT{Corrected $\hat{\upsilon_t}$}
      \STATE $\hat{g_t} \gets \frac{g_t}{1-\prod\nolimits_{i = 1}^{k}\beta^t_2} $ \COMMENT{Corrected $\hat{g_t}$\label{alg:Nadam-Line}}
      \STATE $W_t \gets W_{t-1}-\alpha
                          \frac{\left(\beta^{k+1}_1\hat{m_t}+\left(1-\beta^t_1\right)\hat{g_t}\right)}
                               {\sqrt{\hat{\upsilon_t}}+\epsilon}$
                               \COMMENT{Update parameters\label{alg:Nadam-Line-update}}
      % \STATE $W_t \gets W_{t-1}- \alpha \frac{\hat{m_t}}{\sqrt{\hat{\upsilon_t}}+\epsilon} $ // ??
      \ENDWHILE
    \end{algorithmic}
    \label{alg:nadam}
\end{algorithm}
Line~\ref{alg:Nadam-Line} and~\ref{alg:Nadam-Line-update} add additional calculations involving gradient and parameters update, intending to improve convergence speed.
% \textcolor{red}{Another error. This time with SQRT. According to Adam and source code in TF for Nadam, epsilon is not inside sqrt(). Although the original article stands otherwise, Xiao noticed that too and corrected. I am sticking with whatever I have at TF. There is not $\beta_i$ and not $\beta_2$ in second moment update equation. All of that corrected.}

%
%
The second \mbox{Algorithm~\ref{alg:adamax}} in the ensemble sequence is AdaMax, \mbox{Algorithm~\cite{kingma_adam_2017}}, another modification of the Adam.
\begin{algorithm}\captionsetup{labelfont={sc,bf}, labelsep=newline}
    \caption{Adaptive Moment Estimation based on the infinity norm (Adamax)}
    \begin{algorithmic}[1]
      \STATE \textbf{Number of input samples} \\ $N\gets length(\textit{input data})$\\
      \STATE \textbf{Size of windows} \\ $S\gets length(V_{i..n})$\\
      \STATE Input: $x_n = [V_{i..n}, I_{i..n}, T_{i..n}] - $Shape: $X = (N, S, 3)$
      \STATE Output:$y_n = [SoC_{n}] - $Shape:$Y = (N, 1)$
      \STATE Define Loss function: $L$ \\
             Get hyperparameters: $\alpha, \beta_1, \beta_2, \epsilon$
      \WHILE{$W_t \text{ not converge}$}
      \STATE $t \gets t+1$
      \STATE $g_t \gets \nabla_\phi L_t (W_{t-1})$ \COMMENT{Obtain gradient}
      \STATE $m_t \gets \beta_1 m_{t-1}+(1-\beta_1) g_t $ \COMMENT{$1^{st}$ moment calculation}
      \STATE $\upsilon_t \gets max\left(\beta_2\upsilon_{t-1}, |g_t|\right) $ \COMMENT{Corrected $\hat{\upsilon_t}$ \label{alg:AdaMax-Line}}
      % \STATE $W_t \gets W_{t-1}- \alpha \frac{1}{1-\beta^t_1}\frac{m_t}{\upsilon_t} $
      \STATE $W_t \gets W_{t-1}- \alpha \frac{m_t}{(1-\beta^t_1)(\upsilon_t+\epsilon)} $ \COMMENT{Update parameters}
      \ENDWHILE
    \end{algorithmic}
    \label{alg:adamax}
\end{algorithm}
The second-order moment on Line~\ref{alg:AdaMax-Line} gets replaced with the infinity norm of the moment.
As a result, Xiao et al.~\cite{xiao_accurate_2019} considered AdaMax to have a stable weight updating rule and used in the fine-tuning phase since its' advantage lies in the reduction of gradient fluctuation. 

%
%
Xiao et al.~\cite{xiao_accurate_2019} considered separating the training process into two stages: pre-training and fine-tuning.
Based on his observations: "The purpose of the pre-training phase is to endow GRU model with the appropriate parameters to capture inherent features of the training samples.
The Nadam algorithm uses adaptive learning rates and approximates the gradient using the Nesterov momentum, thereby ensuring fast convergence of the pre-training process."~\cite{xiao_accurate_2019}
The selection of the second algorithm is trivial.
Although, Xiao et al.~\cite{xiao_accurate_2019} selection of AdaMax is defined by fast-convergence to a more stable value for further parameter adjustment.
% \textcolor{red}{I'll be cursed. No, epsilon does. In the tensorflow implementation epsilons added to the vt. Corrected formula added, but commented out. \\
%Remarks made by Xiao justifies the usage of Ensemble algorithm. How would I reprase it into something shorter}\\
%\textit{Remark 1:} The purpose of the pre-training phase os to endow GRU model with the approppriate parameters to capture inherent featires of the training samples. The Nadam algorithm uses adaptive learning rates and approximates the gradient by means of the Nesteriv momentum, thereby ensuring fast convergence of the pre0training process.
%\textit{Remark 2:} The purpose of the fine-tuning phase is to furher adjust the parameters to achieve greater accuracy by means of the AdaMax algorithm, which converges to a more stable value. \\ \\
The proposed Ensamble algorithm combines both methods for single GRU training, \mbox{Algorithm~\ref{alg:ENS}}.
\begin{algorithm}\captionsetup{labelfont={sc,bf}, labelsep=newline}
    \caption{Ensemble optimisation training process}
        \begin{algorithmic}[1]
            % \STATE \textbf{Input:} Data sample with shape=(1,500,1,3) 
            % \STATE \textbf{Output:} Predicted SoC shape=(1,1,1)
            \STATE Setup model. Split total number of epoch by 30\% to $p_{1}$ and $p_{2}$
            \STATE Initialise parameters
            \WHILE{epoch  $<p_{1}$:}
                \IF{epoch $<p_{2}$:}
                    \STATE \COMMENT{pass if already compiled with Nadam}
                    \STATE compile model with Nadam parameters. \COMMENT{pre-training phase}
                \ELSE
                    \STATE \COMMENT{pass if already compiled with AdaMax}
                    \STATE compile model with AdaMax parameters. \COMMENT{fine-tuning phase}
                \ENDIF
                \STATE train for single epoch
            \ENDWHILE
        \end{algorithmic}
    \label{alg:ENS}
\end{algorithm}
The following algorithm describes the adapted version of the Ensemble algorithm, used by the model training procedures, with Nadam for pre-training and AdaMax for fine-tuning phases.
%\textcolor{red}{Descibe what shape has been used. Single sample, history length, batch size (if applicable), features number}

%
%
\mbox{Table~\ref{tab:ensemble-params}} higlights hyper parameters selected by Xiao et al.~\cite{xiao_accurate_2019}.
\begin{table}[htbp]
    \renewcommand{\arraystretch}{1.3}
    \caption{Ensemble, specific parameters}
    \centering
    \label{tab:ensemble-params}
    \resizebox{\columnwidth}{!}{
    \begin{tabular}{ l c c c c }
        \hline\hline \\[-3mm]
        Optimiser     & $\alpha$ & $\beta_1 $ & $\beta_2$ &   $\epsilon$ \\
        \hline
        Nadam
                & $0.001$ & $0.99$ & $0.999$ & $10^{-8}$ \\% 0.0000001
        AdaMax
                & $5*10^{-4}$ & $0.99$ & $0.999$ & $10^{-8}$ \\% $10^{-8}$
        \hline
    \end{tabular}
    }
\end{table}
