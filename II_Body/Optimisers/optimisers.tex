The optimisation algorithm aims to define a way to achieve minimal difference between model prediction and actual values using the mean average error (MAE) equation as a loss function, \mbox{Table~\ref{tab:experiment}}.
The following section breaks down several methods selected by the chosen authors in terms of their growing complexity.
% \subsection{Types of optimisers}
% The optimiser selection defines how fast the model will achieve the local minimum.
Different algorithms have utilised several improvements to achieve an optimum result quicker and avoid overfitting.
However, there is no universal best choice to obtain the best result.
For the state-of-charge estimation, this research intended to attempt multiple algorithms and determine if there was a best choice for the time-series charge cycling problem.
All models shared the same parameter, the learning rate $\alpha$, which acted as a step to update the predictions.
