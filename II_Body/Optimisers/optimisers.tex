The optimisation algorithm aims to define a way to achieve the minimal difference between model prediction and actual values, using the Mean Average Error (MAE) equation as a loss function, \mbox{Table~\ref{tab:experiment}}.
The following section breaks down several methods selected by chosen authors in the growing complexity.
% \subsection{Types of optimisers}
% The optimiser selection defines how fast the model will achieve the local minimum.
Different algorithms utilised several improvements to achieve an optimum result quicker and avoid overfitting.
However, there is no universal best choice to get the best result.
For the State of Charge estimation, the research intended to attempt multiple algorithms and determine if there is the best choice for the time-series charge cycling problem.
All models share the same parameter, the learning rate $\alpha$, which acts as a step to update predictions.

