\subsection{Battery Data for training and validation} \label{subsec:b_data}
Model training has been conducted over Lithium-Ion battery cycling data obtained by the Battery Research Group of the Center for Advanced Life
Cycle Engineering (CALCE) Group at University of Maryland~\cite{noauthor_calce_2017} in 2012.
According to the attached paper, the battery cycling has been conducted with Arbin BT2000 tester machine and controlled with official Arbins Mits Pro Software (v4.27)~\cite{xing_state_2014}.
The SoC value is calculated from the following equation where $C$ represent Charge and $D$ Discharge Capacities in $Ah$.
The result of SoC is equivalent to Coulomb Counting where Nominal Capacity $C_{N}$ converted to seconds, by \mbox{Equation~\ref{eq:SoC-calc}}.
The expected value gets rounded by two decimal places in all scenarios to simplify training and testing processes. 
\textcolor{red}{The value of charge across 25 degrees have not matched with battery testing of the similar cells, conducted during research with freshly bought M1-B series cells.
Despite their similarities, indicating that they have been using in a series of two cells with doubled current or have been through more than 1000 cycles.}
\mbox{Table~\ref{tab:battery}} highlights selected battery characteristics directly from datasheet~\cite{noauthor_anr26650m1a}.
The Battery Cycling data in a Battery testing chamber were stored as Excel spreadsheets over temperature range 0\textdegree{}C and 50\textdegree{}C degrees.
Each testing cycle contains three testing profiles: Dynamic stress test (DST), (US06) and the Federal urban driving schedule (FUDS).
The temperature steps 10 degrees with a tolerance of around 0.5-1 degrees.
\textcolor{red}{The range of 20\textdegree{}C to 50\textdegree{}C was used as a training and validation dataset since this is the most common temperature range, which custom build electric vehicle by QUT motorsport has been experiencing during endurance runs.}
This results in 66763 and 8200 samples for training and validation over the same profile.
Each method went through a single cycling profile and was tested against the other two, as per Mamo et al.~\cite{mamo_long_2020} research.
The final testing has been conducted over two cycles of 25\textdegree{}C and 30\textdegree{}C samples for each of the two remaining profiles, leading to 16510 number of samples.

\begin{table}[ht]
    \renewcommand{\arraystretch}{1.3}
    \caption{Battery characteristics}
    \centering
    \label{tab:battery}
    \resizebox{\columnwidth}{!}{
    \begin{tabular}{ l c c c c c c }
        \hline\hline \\[-3mm]
        Brand & Cell      & Cell & Battery & Nominal          & Nominal & Charge/discharge\\
        name  & Chemistry & Type & Weight  & Capacity $C_{N}$ & Voltage & cut-off voltage \\
        % \makecell{Brand name} & \makecell{Cell Chemistry} & \makecell{Cell Type} & \makecell{Battery Weight} & \makecell{Nominal Capacity $C_{N}$} & \makecell{Nominal Voltage} & \makecell{Charge/discharge cut-off voltage} \\
        \hline
        %A456 \\ (former A123) & 76g+-1g & 2.3Ah & 3.2V & 3.65V, 2.0 V\\
        A123 (2012) & LiFePO4 & ANR26650M1-A & 70g \textpm 2g & 2.3Ah & 3.3V & 3.6V, 2.0 V\\
        \hline\hline
    \end{tabular}
    }
\end{table}
