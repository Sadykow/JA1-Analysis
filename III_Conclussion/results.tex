Two tables, \ref{tab:acc-results1} and \ref{tab:acc-results2}, contains results of accuracy validation on six implemented models over entire drive cycles datasets.
Figures between \ref{fig:Model-1res} and \ref{fig:Model-6res} demonstrated the best-selected cases for visual demonstration and comparison of training on one profile and validation against the other two.
Figures \ref{fig:Model-1losses} to \ref{fig:Model-6losses} refer to the best model over the learning process, based on minimum Mean average or Root Mean Squared errors.

%
%
Despite multiple tests over two cell types, there is no apparent advantage in using LSTM or GRU layers.
To determine the actual performance, it will require multiple trained models over a single implementation to average performance results and make a clear statement.
Models \#1 and \#2 give an illusion of an advantage to one model over another.
However, the accuracy plots in Figures~\ref{fig:Model-1losses} and ~\ref{fig:Model-2losses} indicate how error degrades with time for both models.
The advantage of one over another is a simple matter of randomness in the initial training results.
The attention layer may not significantly boost the training or accuracy, but it gave a good foundation for further improvements and modifications.
Since the SoC estimation is not a pure number based behaviour but also a matter of physical, electrical properties, manual adjustment weights, losses or data itself will not bring valuable results.
Further research and adjustments must be made using a similar principle to improve the training procedure with augmented models. 
For example, the sigmoid function selection in the model output minimises the possibility of going over 100\% or 0\% of charge.

%
%
The best performance with Stateful models can be achieved through using a separate training process for charge and discharge. 
Subfigure~\ref{subfig:Model-4res-DSTtr} demonstrates training over DST, which sufferers from high error in both charge and discharge process.
The other two training were performed with discharge sets only, Subfigures~\ref{subfig:Model-4res-UStr}, \ref{subfig:Model-4res-FUDStr}.
Stateful models can not be validated using traditional means of accurate measurement.
Table~\ref{tab:acc-results2} for Model \#4 takes results from a single cycle only, same as~\ref{fig:Model-4res}, since there is no straightforward way to validate across all cycles (marked with $*$ symbol).

%
%
The advantage of modifying optimisers is better observed on the accuracy plots, Figure~\ref{fig:Model-2losses} and ~\ref{fig:Model-5losses}.
Model \#2 had better results in avoiding overfitting since it used two optimisers for quick adjustment and tuning.
Similar to Model \#5, which had a small learning rate, but modified parameter update with direct involvement of the loss values.
An early termination over \#5 and \#6 is a result of overfitting or apparent stability in the accuracy.
With the number of samples, which the training process went through and based on the RMSE plot, there was a little need for repeated training over the same data repeatedly, as proven in the first several models.

%
%
Overall, there is an obvious advantage in training a model over a single profile and then testing against similar scenarios, for example, creating a model that fits a single drive's driving behaviour over specific driving scenarios.
However, it will suffer from inaccuracies if models will be placed under other conditions without a post-training process with new data.
Comparison of the validation data act as a determined prove.