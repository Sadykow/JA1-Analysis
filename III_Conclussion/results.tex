Since it is a common practice for temperatures on the battery inside EV to spike from 20 ambient degrees to the limit of 60, all temperature ranges were used together to train each model.
The training process was conducted through all datasets for a single battery testing profile and validated on a single cycle of unseen data of 25\textdegree{}C (less or around 20\% of the entire set).
This approach led to the accuracy being lower than other researchers reported by training individually for temp ranges like with Xiao et al.~\cite{xiao_accurate_2019}.
The following section compares the models trained on each individual and then tested against the entire dataset of all three profiles but of a different cell.
Unlike stateful models, all stateless examples were trained using charge and discharge cycles.

%
%
% Final tests for a model performance were conducted against an entire set of two remaining profiles separately.
The metrics were reported using the equations outlined in the \mbox{Table~\ref{tab:metrics}}.
Figures were generated during each iteration of the training process, from the data samples outlined in the \mbox{Subsection~\ref{subsec:b_data}}.
After completing the predefined amount of epochs, each metric was recorded in a comma-separated file to produce accuracy plots, allowing to assess the efficiency of the learning process.

%
%
%%%%%%%%%%%%%%%%%%55
\subsection{Models results overview}
Implementation of several different variations of time-series modells allowed to analysise multiple path of evolment machine learning techniques in State of Charge estimation.
Review of the resulted accurtacy helps make areasonable justifaction for further esearch


Model 1: US06 and FUDS has shown good accuracies, but the training has not been very smooth unlike with DST.
However, the US06 made the best result in capturing both features.
By analysung Table 1, US06 and FUDS can capture each other behavior, but not the DST.
In this sense, DST may seems better feature, but Figure 7.b proves that model des not predict middle flat area.

%
%
Model 2: Adding another layer only added discrepancy in overall prediction and only lead to better but slover training.

%
%
Model 3 Adding attention layer made an improvement on the FUDS training and better fir on Figure 9.i.
Although, it had promessing start by the training accuracyy, this techniques has a mean to maje model better rather than just adding more layers to the model.

%
%
Model 4: Statefull techniques have been a promessing approach but only to very specific scenario.
By itself, it will take long time to achieve accuracy desired.
However, training 2 models sperately for charge and discharge or use it additional technique on short prediction, for example during acceleration events - it has a means for existems.

%
%
Model 5: 1,3 mdeol applied modification to the structure directly and 4th one indirectly. (2nd one did not, read again).
There as (2?) and 5 improves optimisation steps, leading to better accuracy in all testing.
RoAdam allowed faster convergences to the optimal accuracy and generaly achioeved better results by Table 2.

%
%
\subsection{Discussion to add}
Acubs as comparison to model 1.
Separating to multiple layers to the same amount of untis did not lead to improvements.
It was the fastest what achieved the lowest training accuracy, but without affecting learning rate, model could not achieve capturing the other trens.

%
%
This all shown that model not able to capture middle area.
However, combining all 3 techniques into single model may lead to accurate results.

%
%
Model 1,5,6 may be matter of randomness, more that just technique eficiency.
DST generally faster singe profile itsekf is very easy to handle.
+ Model 4 had no means to perform test on entire set of both profiles.
%%%%%%%%%%%%%%%%%%%%
%% PUT both writing and the results from. After you do thatm compare the results as discussion. Expplicitly interpreter that.
%
%
\subsection{******}
\mbox{Two tables, \ref{tab:acc-results1} and \ref{tab:acc-results2}}, contains results of accuracy validation on six implemented models over entire drive cycles datasets.
\mbox{Figures between \ref{fig:Model-1res} and \ref{fig:Model-6res}} demonstrated the best-selected cases for visual demonstration and comparison of training on one profile and validation against the other two.
\mbox{Figures \ref{fig:Model-1losses} to \ref{fig:Model-6losses}} refer to the best model over the learning process, based on minimum Mean average or Root Mean Squared errors.

%
%
Despite multiple tests over two cell types, there is no apparent advantage in using LSTM or GRU layers.
To determine the actual performance, it will require multiple trained models over a single implementation to average performance results and make a clear statement.
Models \#1 and \#2 give an illusion of an advantage to one model over another.
However, the accuracy plots in \mbox{Figures~\ref{fig:Model-1losses} and ~\ref{fig:Model-2losses}} indicate how error degrades with time for both models.
The advantage of one over another is a simple matter of randomness in the initial training results.
The attention layer may not significantly boost the training or accuracy, but it gives a good foundation for further improvements and modifications.
Since the SoC estimation is not a pure number based behaviour but also a matter of physical, electrical properties, manual adjustment weights, losses or data itself will not bring valuable results.
Further research and adjustments must be made using a similar principle to improve the training procedure with augmented models. 
For example, the sigmoid function selection in the model output minimises the possibility of going over 100\% or 0\% of charge.

%
%
The best performance with Stateful models can be achieved through using a separate training process for charge and discharge. 
\mbox{Subfigure~\ref{subfig:Model-4res-DSTtr}} demonstrates training over DST, which sufferers from high error in both charge and discharge process.
The other two training were performed with discharge sets only, \mbox{Subfigures~\ref{subfig:Model-4res-UStr}, \ref{subfig:Model-4res-FUDStr}}.
Stateful models can not be validated using traditional means of accurate measurement.
\mbox{Table~\ref{tab:acc-results2}} for Model \#4 takes results from a single cycle only, same as~\ref{fig:Model-4res}, since there is no straightforward way to validate across all cycles (marked with $*$ symbol).

%
%
The advantage of modifying optimisers is better observed on the accuracy plots, \mbox{Figure~\ref{fig:Model-2losses} and ~\ref{fig:Model-5losses}}.
Model \#2 had better results in avoiding overfitting since it used two optimisers for quick adjustment and tuning.
Similar to Model \#5, which had a small learning rate, but modified parameter update with direct involvement of the loss values.
An early termination over \#5 and \#6 is a result of overfitting or apparent stability in the accuracy.
With the number of samples, which the training process went through and based on the RMSE plot, there was a little need for repeated training over the same data repeatedly, as proven in the first several models.

%
%
Overall, there is an obvious advantage in training a model over a single profile and then testing against similar scenarios, for example, creating a model that fits a single drive's driving behaviour over specific driving scenarios.
However, it will suffer from inaccuracies if models are placed under other conditions without a post-training process with new data.
Comparison of the validation data act as a determined prove.
