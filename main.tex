\documentclass[fleqn,10pt]{olplainarticle}
% Use option lineno for line numbers 

\title{Critical analysis of Lithium Battery State of Charge Estimation using Memory Mechanisms of Machine Learning}

\author[1]{Mr Marat (Matt) Sadykov}
\author[2]{Dr David Holmes}
\affil[1]{Queensland University of Technology}
%\affil[2]{Address of second author}

\keywords{BMS, NN, RNN, LSTM, GRU, SoC}

\begin{abstract}
    \textbf{State of Charge (SoC) estimation is one of the critical parameters, which determines the state at which electrical system currently is.
    Popular in Electrical Vehicle (EV) especially for Formula FSAE competitons etc.
    The articles summarises several methods to estimate SoC using 3 types of Machine Learning (ML). Recurrent Neural Network (RNN), Long-Short Term Memory (LSTM) Neural Networ and Gated Recurrent Unit (GRU) Neural Network.
    It also includes some of the other variations of those algorith to make improvements in model preperation.}
\end{abstract}

\begin{document}

\flushbottom
\maketitle
\thispagestyle{empty}
%%% Introduction
\ifthenelse{\boolean{thesis}}{
    Following the Literature Review, this chapter will analyse the selected branch of Neural Networks in detail.
    It explores the Recurrent NN in depth, and establishes the methodology for further research and implementation 
} {
    %
    % Importance of Electric Vehicles and their battery problem.
    The market for Electrical Vehicles (EVs) has grown significantly over the past decade~\cite{state-ev-australia}.
    The replacement of a fossil fuel-based engine with an electric drivetrain eliminates exhaust emissions with the potential to significantly reduce human impact on climate change.
    For EVs to grow market share and reduce costs, battery cost and longevity must be improved.
    Extensive battery cycling leads to battery degradation over time (ageing).
    The development of smarter and more accurate battery management strategies may be capable of prolonging service duty.
    This would rely upon a system's ability to estimate a battery's state at any point in time.
    An accurate charge calculation avoids overcharging or over-discharging, leading to improved battery service utilisation, better health estimation, longer life span, more reliable range prediction and further benefits~\cite{calif_proper_2008}.
}

%
% The state of Charge, methods and its' value on BMS.
The development of effective methods for State-of-Charge (SoC) estimation remains a topic of crucial research focus.
Various techniques to estimate the SoC have been developed to enhance battery usage.
The ability to determine the state of a battery or a battery system is a required function for an advanced Battery Management System (BMS).
Those techniques can be classified into three primary categories~\cite{ali_towards_2019,ng_enhanced_2009,robust_SoC,6953745}: direct measurement, model-based methods, and computer intelligence or Machine Learning (ML).
Direct measurement methods take readings from batteries, relying on sensors such as open circuit Voltage, internal resistance, or current readings over set periods (i.e. Coulomb Counting)~\cite{ng_enhanced_2009,robust_SoC}.
Model-based methods recreate battery behaviour and use sensor inputs to calculate results from a pre-defined model~\cite{6953745}.
Computer intelligence techniques enhance such models with additional data.
Those data-driven calculations aim to improve model estimation by fitting to an actual behaviour observation.
Examples include Fuzzy Logic~\cite{malkhandi_fuzzy_2006}, Support Vector Machine~\cite{hansen_support_2005, anton_battery_2013}, or Neural Networks (NN)~\cite{song_lithium-ion_2018,Chemali2017,mamo_long_2020,jiao_gru-rnn_2020,xiao_accurate_2019,javid_adaptive_2020,zhang_deep_2020}.

%
% Comparison between traditional methods and ML-based estimations.
While some model-based methods, such as the equivalent circuit model, are simple to implement within a BMS, many cannot correctly capture a battery's complex multi-dependant behaviour~\cite{6953745}.
%Others are limited to offline simulation due to high complexity and computation requirements, therefore, not suited to an onboard BMS.
Direct measurement estimation is limited to sensor accuracy and is affected by losses created by Coulombic efficiencies~\cite{Smith_2010}, where some portion of charge gets transferred to heat or is affected by battery ageing that is not captured.
%Additionally, the measured voltage depends on whether the battery is in use or has been on rest for a given time.
%An accurate SoC estimation model has to implement a way to accommodate battery losses and sensor readings.
%If model-based methods treat as constant values, then in real scenarios, they can be significantly limited to sensor inaccuracy, environmental and internal battery temperature, or ageing.
%In contrast, Neural Network methods can accommodate losses as they can capture complex phenomenological behaviours~\cite{bengio_learning_1994}.
%
%Smart BMSs, which incorporate some method of ML, use only sensory data~\cite{zhang_deep_2020} and do not require batteries' physical property~\cite{zhang_deep_2020}.
In contrast, Machine Learning can establish relationships in complicated and multi-dimensional non-linear systems~\cite{hansen_support_2005,anton_battery_2013,he_state_2014}.
This characteristic shows excellent potential to account for battery losses due to Coulombic efficiency.
Some researchers used the support vector machine-based methods to estimate SoC using voltage, current, and temperature inputs~\cite{hansen_support_2005,anton_battery_2013}.
Sensory data was obtained from a driving schedule profile on a battery cycler, and the end error estimation was achieved as less than 6\%~\cite{he_state_2014}.
%The solution to lowering that percentage may lie within the more complicated models, like Neural Networks.
Many attempts to implement different Neural Networks exist, but the most promising variant for charge estimation is Recurrent Neural Networks (RNN)~\cite{song_lithium-ion_2018, Chemali2017, mamo_long_2020, jiao_gru-rnn_2020, xiao_accurate_2019, javid_adaptive_2020, zhang_deep_2020}.
The effectiveness of RNNs in time-series dependant problems has been shown using internal neurons to process data sequences with varying lengths by Chemali \textit{et al.}~\cite{Chemali2017}.
% The models' cells act as memory units, building relationships and giving outputs based on multiple inputs over time.
% Typical examples are data forecasting, handwriting, speech and image recognition, machine translation or music composition~\cite{devdarshan_applications_2019}. 

% 
% What has been done in the area of RNN for battery characterisation
Over the past five years, the RNN approach has found multiple applications for SoC estimation.
The earliest approach utilised a battery charge's regression nature only using stateless models~\cite{song_lithium-ion_2018,jiao_gru-rnn_2020,xiao_accurate_2019}.
%\textcolor{red}{when input impact is not preserved for further predictions}.
% non-connected predictions, allowing to be used at any time, independently from previous results or inputs.
Later, some approaches introduced additional parameters to support the NN learning process~\cite{mamo_long_2020,jiao_gru-rnn_2020,javid_adaptive_2020}.
Besides good convergence, these models can determine critical events, like the time before complete charge depletion or overcharge.
However, wide application has been limited due to the need for the initial state as an input feature.
The most popular approach determines the charge's value using a fixed-size recent history of voltage, current, and temperature in stateless Long Short-Term Memory (LSTM) models~\cite{Chemali2017,mamo_long_2020,javid_adaptive_2020,zhang_deep_2020}.
This method has the advantage of being independent of the charge or discharge cycles at different periods, as long as the history samples are in an equally time-spaced order.
%However, estimation determines values based on fixed history samples and does not preserve for the next prediction, making practical estimation of single or several charge values, but not determining critical events ahead.
The most recent attempt to determine the Li-Ion battery's remaining useful life was with the implementation of the Gradient Recurrent Unit (GRU) models~\cite{song_lithium-ion_2018,javid_adaptive_2020,xiao_accurate_2019,jiao_gru-rnn_2020}, when every prediction was independent from previous, allowing it to be used at any random point of time, without worrying if a battery was initially fully charged or depleted.
% An alternative method could be with stateful models, which continuously preserve the prediction's impact and can be propagated further until they reach the end state without resets~\cite{zhu_statefulnes_tfdocs_2020}.
While it applies to the estimation of regenerative braking, the stateful models are more applicable for a critical event time estimation, like a prediction of the remaining batteries' life.
% \mbox{Table~\ref{tab:review}} summarises the range of methods applied to SoC estimation, highlighting the model cell type, structure to define input sample type, optimiser selection, and additional features introduced by authors to improve predictions.
Focusing specifically on RNN models applied to SoC estimation, \mbox{Table~\ref{tab:review}} presents a range of such methods developed in recent years.

%
% Intro where are specifics. Hyper-para. THere is w\a hole range, Table summarises many.
The earliest attempts to train an RNN model to predict the SoC was to fit several cycles of a single battery utilisation dataset at different temperatures~\cite{song_lithium-ion_2018, xiao_accurate_2019,javid_adaptive_2020, jiao_gru-rnn_2020}.
Later, it was used to generalise battery behaviour to multiple usage scenarios, leading to higher Root Mean Squared Error, and broader applications, like in Mamo and Wang's~\cite{mamo_long_2020} work.
%For instance, by comparing similar procedures of testing from Song et al.~\cite{song_lithium-ion_2018} and Mamo et al.~\cite{mamo_long_2020} who conducted their research methods but used different validation mechanisms, it can be seen how accuracy error doubles if testing performs at a different temperature or untrained driving profile.
This approach led to doubled accuracy error on the testing data, performed at a different temperature or untrained driving profile, as compared with similar testing procedures from Song \textit{et al.}~\cite{song_lithium-ion_2018}, and Mamo and Wang~\cite{mamo_long_2020}.
% By selecting the same drive cycles at a closely similar temperature, Song et al.~\cite{song_lithium-ion_2018} accuracy achieved roughly 0.735\% error, then as Mamo et al.~\cite{mamo_long_2020} reported 1.2533\% at best.
Doubling the amount of data by combining several temperatures or profiles also showed insignificantly higher error but improved general capture, as per stateful models in Song \textit{et al.}~\cite{song_lithium-ion_2018} with roughly 0.735\%, and stateless Mamo and Wang~\cite{mamo_long_2020} reporting 1.2533\%  errors respectively.
Those numbers can be explained by using a single driving profile but with an entire available temperature range for training, and then a portion of handpicked temperatures for validation and accuracy reporting.
Such an approach does not necessarily represent a realistic EV usage scenario well since, during a single acceleration event, the battery goes through ambient temperature to the maximum allowed within several seconds, and its' usage depends on the road conditions and the driver's experience.
One of the potential ways to improve that capture is to modify the structure of the models, introducing an additional layer of logic, like Attention, as per Mamo and Wang~\cite{mamo_long_2020}, or Extra Dense layers, as per Jiao \textit{et al.}~\cite{jiao_gru-rnn_2020}, making a model applicable to any driving condition.
% There has been little research to validate the performance of different Machine Learning techniques to extrapolate ideal laboratory battery cycling conditions of early collected data, to an electric vehicle behaviour.
%% (Include one of the sections)
% After identifying the most optimal input preferences without significant effect on the performance, one of the further developments applies changes to the structure of GRU or LSTM by adding additional layers (Attention or Extra Dense layers)~\cite{mamo_long_2020, jiao_gru-rnn_2020}.
% Another way is to change the optimisation process to achieve similar accuracy faster (i. e., adding a Momentum algorithm to the Stochastic Gradient Optimisations process or replacing gradient calculations with statistical)~\cite{xiao_accurate_2019, javid_adaptive_2020}.
% Those methods modify standard ways introduced earlier in model training by applying additional operations.
% As a result, they met to achieve faster accuracy using similar training approaches. 
Another strategy would be to use a variety of statistical or gradient-based optimisers (i.e., adding a Momentum algorithm to the Stochastic Gradient Optimisations process~\cite{xiao_accurate_2019}) to speed up the training or extra multiple potential minimal, achieving the lowest error or identifying the most suited for a given scenario.
Due to the stochastic nature of ML, it is hard to give any clear winner among optimisers by only judging their complexity, not average performance with multiple trials.
% \textcolor{red}{\textbf{Matt: I moved most of the details about the table to caption. However, since this paragraph and the next were swapped, I had to do my best to avoid driving profile mentioning where possible. Perhaps we should consider swapping them back or some sentences.}}
\ifthenelse{\boolean{thesis}}{
\begin{table}[h]
    \renewcommand{\arraystretch}{1.3}
    \caption{Evaluated papers implementation summary.
    The model type highlights a primary path in structuring a Neural Network.
    Statefulness defines the input method, where stateless uses a fixed size of input samples per feature and statefully applies each time-sample one at a time in batches.
    %Optimiser selection sets the algorithm for the learning process from one of the following methods: Adaptive moment estimation (Adam), Nesterov adaptive moment estimation  (Nadam), Stochastic gradient descent (SGD), AdaMax (AM) and Differential Evolution (DE).
    Optimisers are defined from Adaptive moment estimation (Adam), Nesterov adaptive moment estimation  (Nadam), Stochastic gradient descent (SGD), AdaMax (AM) and Differential Evolution (DE).
    }
    \centering
    \label{tab:review}
\resizebox{\textwidth}{!}{
\begin{tabular}{l|c|c|c|c|c|c|c|c|c|l}
    \hline\hline \\[-4mm]
    \multicolumn{1}{ c }{} & 
    \multicolumn{2}{|c|}{Model} & 
    \multicolumn{2}{ c|}{State-} & 
    \multicolumn{5}{ c|}{Optimiser} &
    \multirow{2}{ 4em }{Extension} \\
    \cline{1-4} \cline{5-10}
    Reference source & GRU  & LSTM & -less & -ful & Adam & Nadam & SGD & AdaMax & DE\footnote{Differential Evolution} & \\
    \hline
    Song \textit{et al.}~\cite{song_lithium-ion_2018}
        & \chk &      &      & \chk & \chk &      &      &      &      & 4 Layers  \\
    Chemali \textit{et al.}~\cite{Chemali2017}
        &      & \chk & \chk &      & \chk &      &      &      &      &           \\
    Mamo and Wang~\cite{mamo_long_2020}
        &      & \chk & \chk &      &      &      &      &      & \chk & Attention \\
    Jiao \textit{et al.}~\cite{jiao_gru-rnn_2020}
        & \chk &      &      & \chk &      &      & \chk &      &      & Momentum  \\
    Xiao \textit{et al.}~\cite{xiao_accurate_2019}
        & \chk &      &      & \chk &      & \chk &      & \chk &      & Ensemble  \\
    Javid \textit{et al.}~\cite{javid_adaptive_2020}
        & \chk &      & \chk &      & \chk &      &      &      &      & Robust    \\
    Zhang \textit{et al.}~\cite{zhang_deep_2020}
        &      & \chk & \chk &      &      & \chk &      &      &      & Online    \\
    \hline\hline
\end{tabular}
}
\end{table}
} {
\begin{table}[H]
    \renewcommand{\arraystretch}{1.3}
    \caption{Evaluated papers implementation summary.
    The model type highlights a primary path in structuring a Neural Network.
    Statefulness defines the input method, where stateless uses a fixed size of input samples per feature and statefully applies each time-sample one at a time in batches.
    %Optimiser selection sets the algorithm for the learning process from one of the following methods: Adaptive moment estimation (Adam), Nesterov adaptive moment estimation  (Nadam), Stochastic gradient descent (SGD), AdaMax (AM) and Differential Evolution (DE).
    Optimisers are defined from Adaptive moment estimation (Adam), Nesterov adaptive moment estimation  (Nadam), Stochastic gradient descent (SGD), AdaMax (AM) and Differential Evolution (DE).
    }
    \centering
    \label{tab:review}
\begin{adjustwidth}{-\extralength}{0cm}
    \newcolumntype{C}{>{\centering\arraybackslash}X}
    \begin{tabularx}{\fulllength}{l|c|c|c|c|c|c|c|c|c|l}
        % \hline\hline \\[-5mm]
        \toprule \\[-6mm]
        \multicolumn{1}{ c }{} & 
        \multicolumn{2}{|c|}{\textbf{Model}} & 
        \multicolumn{2}{ c|}{\textbf{State-}} & 
        \multicolumn{5}{ c|}{\textbf{Optimiser}} &
        \multirow{2}{ 4em }{\textbf{Extension}} \\
        \cline{1-4} \cline{5-10}
        \textbf{Reference source} & \textbf{GRU}  & \textbf{LSTM} & \textbf{-less} & \textbf{-ful} & \textbf{Adam} & \textbf{Nadam} & \textbf{SGD} & \textbf{AdaMax} & \textbf{DE\textsuperscript{1}} & \\
        \hline
        Song \textit{et al.}~\cite{song_lithium-ion_2018}
            & \chk &      &      & \chk & \chk &      &      &      &      & 4 Layers  \\
        Chemali \textit{et al.}~\cite{Chemali2017}
            &      & \chk & \chk &      & \chk &      &      &      &      &           \\
        Mamo and Wang~\cite{mamo_long_2020}
            &      & \chk & \chk &      &      &      &      &      & \chk & Attention \\
        Jiao \textit{et al.}~\cite{jiao_gru-rnn_2020}
            & \chk &      &      & \chk &      &      & \chk &      &      & Momentum  \\
        Xiao \textit{et al.}~\cite{xiao_accurate_2019}
            & \chk &      &      & \chk &      & \chk &      & \chk &      & Ensemble  \\
        Javid \textit{et al.}~\cite{javid_adaptive_2020}
            & \chk &      & \chk &      & \chk &      &      &      &      & Robust    \\
        Zhang \textit{et al.}~\cite{zhang_deep_2020}
            &      & \chk & \chk &      &      & \chk &      &      &      & Online    \\
        % \hline\hline
        \bottomrule
    \end{tabularx}
\end{adjustwidth}
\noindent{\footnotesize{\textsuperscript{1} Differential Evolution}}
\end{table}
}

% Although experiments which involved extrapolation of ideal laboratory conditions with accurate sensory to real-world conditions, worth different environmental characteristics or sensory interference - are not currently available.
%
In most published testing of ML methods applied to SoC, experiments on battery cycling data were conducted on different cell types.
Most used table data of real-time sensory results from battery cyclers to validate efficiencies generated using different current schedulers (driving profiles)~\cite{Chemali2017,song_lithium-ion_2018,mamo_long_2020,xiao_accurate_2019}.
%An equally time-based sample of current consumption acts as an input to the battery cycler, intended to recreate a stress test on a battery or human driving behaviour.
Three profiles are most commonly used in the research in this area: Dynamic Stress Test (DST) for a variable power discharge mode, aggressive Highway Drive Schedule (US06), and Federal-Urban Driving Scheduler (FUDS) for nominal driving scenarios~\cite{xiao_accurate_2019,javid_adaptive_2020,mamo_long_2020}.
Unlike some general simple static discharge processes, which commonly appear in other battery-based tools, driving profiles include some amount of regenerative driving to simulate the actual application of the battery for an Electric Vehicle.
Differences of such drive cycle data in training and testing of Machine Learning SoC estimation have been used including applications focusing the fitting process on battery discharge only~\cite{song_lithium-ion_2018,mamo_long_2020,jiao_gru-rnn_2020,javid_adaptive_2020}; capturing the complete charge-discharge cycle~\cite{Chemali2017}; multiple combinations at various temperatures or profiles~\cite{xiao_accurate_2019,mamo_long_2020,Chemali2017,javid_adaptive_2020}; the impact of data samples' ammount~\cite{song_lithium-ion_2018}; and cross-validation of all three current profiles against each other~\cite{mamo_long_2020}.
% There were several applications for those data, involving focusing the fitting process on battery discharge only~\cite{song_lithium-ion_2018,mamo_long_2020,jiao_gru-rnn_2020,javid_adaptive_2020}, capturing complete charge-discharge cycle~\cite{Chemali2017}, multiple combinations at various temperatures or profiles~\cite{xiao_accurate_2019,mamo_long_2020,Chemali2017,javid_adaptive_2020}, the impact of data samples ammount~\cite{song_lithium-ion_2018} and cross-validation of all three current profiles against each other~\cite{mamo_long_2020}.
%
% Mamo et al.~\cite{mamo_long_2020} conducted experiments by validating the RNN model performance by training one driving profile and testing against the other two.
% The results showed doubled higher Root Mean Squared Error, as opposed to training and validation over single driving profiles as per experiments by Xiao et al.~\cite{xiao_accurate_2019}.
%
% \textcolor{red}{Follow the enumaeration}
% \begin{enumerate}
%     \item only discharge~\cite{song_lithium-ion_2018,mamo_long_2020,jiao_gru-rnn_2020,javid_adaptive_2020}
%     \item charge-discharge~\cite{Chemali2017}
%     \item multiple of cycles, like temps or profiles~\cite{xiao_accurate_2019,mamo_long_2020,Chemali2017,javid_adaptive_2020}
%     \item Limits of one cycle, benefits of multiple or length of it,  examples to compare to~\cite{mamo_long_2020,song_lithium-ion_2018}
% \end{enumerate}
Identifying the best-suited method for a specific condition, like driving an EV, is one of the crucial steps for machine learning engineering.
It requires a carefully defined methodology, which characterises researched conditions as close as possible, and experimental results from multiple models with applicable techniques and the lowest errors.
By comparing implementation and results from different sources and testing accuracy and performance against multiple driving conditions at various temperatures from ambient to maximum possible, it is possible to select the best machine learning technique, which can be integrated directly into an Electric Vehicle and safely used either on tight city roads or long high-speed highways.
%There have not been many similar experiments against other SoC estimation methods since such validation procedures are necessary for the computer intelligence method compared to battery modelling.
% In addition, all investigations did poor research in extrapolating ideal laboratory conditions with battery cycler, stable temperatures and no sensor discrepancies to an actual driving situation on the road, mainly due to lack of car data.
%
% Usage of a single profile's battery cycling data may not validate ML methods' efficiency in driving an electric vehicle.
% The inability to determine the charge's current state during EV driving makes the online learning process inapplicable.
% Even with other SoC estimation technique usage, the computational complexity of training any NN is complicated to fit on a low-power device.
% An offline-trained model had the advantage of insignificant resource consumption during the prediction stage, making it a prefered way for an EV.
% All further model testing will be applied through varying battery cycling profiles to capture the influence of data type to model efficiency.
%
% (No point)
% The drawback lies within the validation procedures of produced models.
% Ideally, this process must be performed with similar data but under different conditions or at different times.
% However, once the produced model is taken through different unseen scenarios, the accuracy lowers drastically compared to early reported results since it starts to experience something completely unexpected.
% The lack of training samples or means to produce at different conditions affected the final judgment results of many publications. 
% Even though the estimation produces accurate output with tabled data, placed under actual driving conditions - the model will have difficulties matching training performance. 


%
%
% Mamo et al. conducted experiments
%
% ------------------------------------------------------------------------ \\
% PAPER CONTRIBUTION may be worth adding as a clear statement \\
% ------------------------------------------------------------------------ \\
% Asses handpicked articles from different categories of SoC estimation, asses efficiency and complexity to utilise in EV.
% This paper's contribution lies in researching recent approaches used to estimate Charge's State from battery sensor readings.
% Comparing and implementing the most promising algorithms intended to determine the direction to build upon in the future. \\ \\
% This paper will test recent developments in the most common SoC prediction to determine the most promising direction/approach towards implementing an improved light model applied to an electric vehicle.
% It determined to find the ML model creation's weakness and implement a new one for an EV. \\
% ------------------------------------------------------------------------ \\
%
%
% This paper investigates, implements, and compares extended memory-based models of RNN to predict the State of Charge and additional built-on over-time techniques to select the most effective and least resource-requiring onboard-based computations applicable within Electrical Vehicles.
\ifthenelse{\boolean{thesis}}{
    This chapter investigates, implements, and compares extended memory-based models of RNN to predict the State of Charge and additional built-on over-time techniques to select the most suitable practical application for EV use with combinations of different profiles.
} {
    This paper investigates, implements, and compares extended memory-based models of RNN to predict the State of Charge and additional built-on over-time techniques to select the most suitable practical application for EV use with combinations of different profiles.
}
Each subset will contain implementation from various key references, changing either structure of the models or learning approaches.
It should help develop a methodology to further extrapolate offline trained methods from the lab condition to the road drive tests.
% The recent advancement and commonly used subsets are Gradient Recurrent Unit and Long Shot-Term Memory unit cells.
% Recurrent Neural Network has been confirmed suitable for the battery-related system by authors discussed further, such as Chemali~\cite{LSTM_Hochreiter1997}.
% However, there is no valid proof of if GRU or LSTM is helpful for battery SoC estimation.
% The best way to separate them and explore differences and efficiencies is to use them as Stateful and Stateless models.
The A123 Lithium-Ion battery data with three typical driving profiles, obtained from the University of Maryland 2012~\cite{noauthor_calce_2017} cycling experiment, will act as training and testing samples.
Each method will be validated through those samples (either DST, US06, or FUDS driving profiles), and tested against the robustness and accuracy of estimation of the State of Charge of batteries in the other two unseen schedulers.

%
%
\ifthenelse{\boolean{thesis}}{
    Since there has been no comparison between which RNN type or driving profile impacts State of Charge estimation for both charge and discharge cycles, this research aims to identify the most viable and optimum method for custom-build Electric Vehicles.
} {
    Since there has been no comparison between which RNN type or driving profile impacts State of Charge estimation for both charge and discharge cycles, this article aims to identify the most viable and optimum method for custom-build Electric Vehicles.
}
% Sometimes, multiple published models presented results relying only on discharge cycles, describing the efficiency of modelling on a single battery or in an ideal lab environment~\cite{xia_state_2018,javid_adaptive_2020,jiao_gru-rnn_2020,mamo_long_2020,zhang_deep_2020}.
However, long overnight charge cycles and regenerative breaking burst charges are equally crucial for the SoC percentage in the context of Electric Vehicles' battery utilisation at prolonged usage and influence models' weight and biasses.
% \textcolor{red}{\textit{
This article will focus on the accuracy of SoC prediction based on the model training from charge and discharge cycles across various temperatures.
% (2)This article will present a novel methodology for comparing multiple models with different techniques to each other, accommodating MLs' stochastic nature.
% (3)This article will compare multiple works in SoC estimation to identify the most applicable to car driving and prose methods to build on top further.
% }}
% ~\cite{Chemali2017,}
%
%
The remaining sections are organised as follows.
\ifthenelse{\boolean{thesis}}{
    Details on algorithms and optimisers are written in Section~\ref{sec:AN:Body}, where Subsection~\ref{subsec:structure} separates all details for each GRU and LSTM method, and Subsection~\ref{subsec:optimisers} breaks down every applied optimising algorithm.
    Applied methodology with details regarding training procedure and hyper-parameters selection are outlined in Section~\ref{sec:Meth}, with processing data in Subsection~\ref{subsec:b_data}.
    Section~\ref{sec:AN:Results} gives the results of implementation and performance characteristics and concludes the critical analysis, while Section~\ref{sec:AN:Conclussion} concludes the Chapter.
} {
    Details on algorithms and optimisers are written in Section~\ref{sec:Body}, where Subsection~\ref{subsec:structure} separates all details for each GRU and LSTM method, and Subsection~\ref{subsec:optimisers} breaks down every applied optimising algorithm.
    Applied methodology with details regarding training procedure and hyper-parameters selection are outlined in Section~\ref{sec:Meth}, with processing data in Subsection~\ref{subsec:b_data}.
    Section~\ref{sec:conclussion} gives the results of implementation and performance characteristics and concludes the critical analysis.
}
%%% Body
\subsection{Recurrent Neural Network (RNN)}\label{sec:RNN}
    Implementation of the model based on simple RNN using multi layer Dense(X) networks.~\cite{lees2010theoretical}
    A very basic version of Recurrent Neural Network consist of very basic layers with some amount of neurons.
    Each neuron uses a single function as an activation one.
    This layer is also refereed to as a Dense Layer.
    There are few possible activation functions which commonly used in Machine Learning Libraries for \textbf{Data Driven} or Time Series questions:
    \begin{itemize}
        \item \textit{linear} - Simple Linear function
        \item \textit{elu} - Exponential Linear function
        \item \textit{relu} - Rectified Linear unit function
        \item \textit{sigmoind} - Sigmoid function $sigmoid(x) = 1/1(1+exp(-x)$
        \item \textit{tanh} - Hyperbolic Tangent function $tanh(x) = (exp(x)-exp(-x))/(exp(x)+exp(-x)))$
    \end{itemize}
    Each Dense layer consist of some amount of neurons, which utilizes a single activation functions.
    To apply multiple different without overcompicating neurons a multiple bunch of layers with different or the same number of neurons or activations functions gets applied. The more layer network contains, the deeper network becomes. They also called as Deep Neural Networks.
    Figure \textbf{N} provides an example of such network.
    Due to inability to directly interact with those layers, they also referred as Hidden layers, where as Input and Output are defined by use and not always fixed. \\
    \textbf{FIGURE OF DEEP NN taken from old articles}


% \subsubsection{Implementation}
%     The input data for a network has been created using Windowing technique, where \textbf{216k} sample of battery data, consisting of State Of Charge only were separated on 500 sample windows.
%     As a results, model outputs a single sample as SoC at next Time Step. Using Tensorflow library and calculating number of Neurons using recomended formula \\
%     \textbf{THIS IS THE BEST PLACE FOR IT}. No other places suites as much.
%     The strcuture of the model has the following form. (Few Dense Layers+Dropount).
%     A Dropout layer has been used to prevent data overfitting. The selection of activation functions has been done through the properties of the data, which model has to fit in and multiple trials.
% \subsubsection{Observation}
%     A simple Recurrent Neural Network has proven itself effective with simple Linear problems. However, with battery state of charge it is unable to capture complicated features like transition between Discharge and Charge or process of Constant-Voltage Constant-Current charging. In the application of battery utilisation inside Electrical Vehichle, this approach can be used only with some additional logic, such as Kalman Filters.
%     The best approach is to introduce more information about battery state and use more complicated version of Time Series capable to memorise feature with time, such as LSTM~\ref{sec:LSTM} and GRU~\ref{sec:GRU}.
%     The results of the prediction discussed in Section~\ref{sec:results}.
    
\section{Long-Short Term Memory (LSTM)}\label{sec:LSTM}
\subsection{Definition}
    Currently, the most common usage of the Time-series Machine LEarning model is the prediction of stock prices, weather prognosition or any other time dependant data.
    However, the most common problem for any of those scenarios is vanishing gradient.
    Long range data tend to fade away from the model, which impacts overall prediction.
    The Long Short-Term Memory (LSTM) models tend to capture long-term \textbf{dependancies by longer preserving the sequence} of data. \\
    The structure of LSTM Recurrent Neural Network is similar to clasical RNN, discussed earlier.
    The difference is withing cells themself.
    Unlike Dense layers with single activation function, the structure LSTM, presented on \textbf{Figure} consists of multiple gates.
    \textbf{Try put very brief desciption from my notebook on structure followed by formulas.} \\ [2 pc]
\subsection{Implementation}
    Following implementation is based on Chemali2017 article, which used similar setup and dataset to predict State of Charge with different Network depts \footnote{history sizes}.
    Following table higlighs parameters, which provided best results for their experements.
    \textbf{Table of the parameters.}
    The network will be the one discussed above.
    Model itself will be multi-feature based with follwoing parameters: Voltage \textit{V(t)}, Current \textit{A(t)} and Temperature \textit{C(t)}, where \textit{t} represents a time-stamp. Each feature will contain equal amount of sample and each will be feed in input column vector. As a result, a single input will have following form: \\
    % [V(0)], [I(0)], [T(0)] \\
    % [V(1)], [I(1)], [T(1)] \\
    % [V(n)], [I(n)], [T(n)] \\
    where \textit{n} is the history size.
    The output vector will be a State of Charge \textit{SoC(\%)} percentage up to 2 decimal places, within range 0 to 1 and time stamp \textit{n}: \\
    % [SoC(n)] \\
    As a result the shape of input and output data will be: X(0)=(n,3) and Y(0)=(1)
    The entire dataset will use single-step windowing tecnhinue with no bathces to save memory and utilsase \textbf{stateless}* \footnote{Need to discuss this with Holmes.} model. \\
    The Generated dataset of sample size \textit{k}, will consist of two Tensor input/output Vectors of following shape: \\
    X = (k-n,n,3), Y = (k-n,1).
    The variable type for computation was selected to be float32.
    To keep Denerated dataset simple, no batching approach spared from dealing with 4-dimensional input vectors.
\subsection{Prediction results}


    Implementation of the model based on Chemali2017. Application for our section and results. Refer to methodology from time to time.
\section{Gradient Recurrent Unit (GRU)}\label{sec:GRU}
Implementation of the model based on BinXio or someone else. Introduce their way of implementation how to improve optimisation using 2 algorithms as per their discussion.
\section{LSTM with Attention}\label{sec:lstm-attention}
Implementation of the model based on TadeleMamo 2020 as the most recent and how to implement improvement to the model to make it better. Gracefully transition idea from here to my model, it is similar.
%%% Conclusion
\section{Conclusion}\label{sec:conclussion}
The work offered several implementation of Machine Learning algorithms for State of Charge estimation for A123 batteries.
\section{Acknowledgements}\label{sec:acknowledgements}
The research was undertaken through funding from the Automotive Engineering Graduate Program (AEGP) in cooperation with the Queensland government and Prohelion industry partner.
All models evaluation has been performed through the help of the QUT HDR research and technical staff, who arranged access to the Hight Performance Machine (HPC Lyra) for the extensive initial computations.
The research has been conducted in cooperation with Mr Sam Haines, who performed a similar investigation on a similar dataset using Kalman filters and its' branches.
In addition, both Sam and I owe our progress to our Principal and Associate supervisors, Associate Professors David Holmes and Geoff Walker, who have been following our progress through the entire research.
Finally, we gratefully acknowledge the original developers of the Machine Learning framework Tensorflow from Google, who wrote detailed guides and documentation for all necessary tools used throughout the investigation.
%It is worth acknowledging the input from Dr Olga, Dr Mahsa and Pr Lovell who showed a practical interesr in the research and ... contribtution.



% \section*{Methods and Materials}

% Guidelines can be included for standard research article sections, such as this one.

% \section*{Some \LaTeX{} Examples}
% \label{sec:examples}

% Use section and subsection commands to organize your document. \LaTeX{} handles all the formatting and numbering automatically. Use ref and label commands for cross-references.

% \subsection*{Figures and Tables}
    
%     Use the table and tabular commands for basic tables --- see Table~\ref{tab:widgets}, for example. You can upload a figure (JPEG, PNG or PDF) using the project menu. To include it in your document, use the includegraphics command as in the code for Figure~\ref{fig:view} below.
    
%     % \begin{figure}[ht]
%     %     \centering
%     %     \includegraphics[width=0.7\linewidth]{frog}
%     %     \caption{An example image of a frog.}
%     %     \label{fig:view}
%     % \end{figure}
    
%     \begin{table}[ht]
%         \centering
%         \begin{tabular}{l|r}
%             Item & Quantity \\\hline
%             Candles & 4 \\
%             Fork handles & ?  
%         \end{tabular}
%         \caption{\label{tab:widgets}An example table.}
%     \end{table}

% \subsection*{Citations}

%     LaTeX formats citations and references automatically using the bibliography records in your .bib file, which you can edit via the project menu. Use the cite command for an inline citation, like \cite{lees2010theoretical}, and the citep command for a citation in parentheses \citep{lees2010theoretical}.

% \subsection*{Mathematics}

%     \LaTeX{} is great at typesetting mathematics. Let $X_1, X_2, \ldots, X_n$ be a sequence of independent and identically distributed random variables with $\text{E}[X_i] = \mu$ and $\text{Var}[X_i] = \sigma^2 < \infty$, and let
%     $$S_n = \frac{X_1 + X_2 + \cdots + X_n}{n}
%           = \frac{1}{n}\sum_{i}^{n} X_i$$
%     denote their mean. Then as $n$ approaches infinity, the random variables $\sqrt{n}(S_n - \mu)$ converge in distribution to a normal $\mathcal{N}(0, \sigma^2)$.

% \subsection*{Lists}

%     You can make lists with automatic numbering \dots

% \begin{enumerate}[noitemsep] 
%     \item Like this,
%     \item and like this.
% \end{enumerate}
% \dots or bullet points \dots
% \begin{itemize}[noitemsep] 
%     \item Like this,
%     \item and like this.
% \end{itemize}
% \dots or with words and descriptions \dots
% \begin{description}
%     \item[Word] Definition
%     \item[Concept] Explanation
%     \item[Idea] Text
% \end{description}

% \section*{Acknowledgments}

%     Additional information can be given in the template, such as to not include funder information in the acknowledgments section.

\bibliography{sample}

\end{document}