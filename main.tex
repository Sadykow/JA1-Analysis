\documentclass[fleqn,10pt]{olplainarticle}
% Use option lineno for line numbers 

\title{Critical analysis of Lithium Battery State of Charge Estimation using Memory Mechanisms of Machine Learning}

\author[1]{Mr Marat (Matt) Sadykov}
\author[2]{Dr David Holmes}
\affil[1]{Queensland University of Technology}
%\affil[2]{Address of second author}

\keywords{BMS, NN, RNN, LSTM, GRU, SoC}

\begin{abstract}
    \textbf{State of Charge (SoC) estimation is one of the critical parameters, which determines the state at which electrical system currently is.
    Popular in Electrical Vehicle (EV) especially for Formula FSAE competitons etc.
    The articles summarises several methods to estimate SoC using 3 types of Machine Learning (ML). Recurrent Neural Network (RNN), Long-Short Term Memory (LSTM) Neural Networ and Gated Recurrent Unit (GRU) Neural Network.
    It also includes some of the other variations of those algorith to make improvements in model preperation.}
\end{abstract}

\begin{document}

\flushbottom
\maketitle
\thispagestyle{empty}
%%% Introduction
\IEEEPARstart{T}{he} market for Electrical Vehicles (EV) has grown significantly over the past decade~\cite{state-ev-australia}.
The replacement of a fossil fuel-based engine with an electric drivetrain eliminates exhaust emissions with the potential to reduce human impact on climate change significantly.
For EVs to grow market share and reduce costs, battery cost and longevity must be improved.
Extensive battery cycling leads to battery degradation over time (aging).
The development of smarter and more accurate battery management strategies may be capable of prolonging service duty.
This would rely upon a system's ability to estimate a battery's state at any point in time.
An accurate charge calculation avoids overcharging or over-discharging, leading to improved battery service utilisation, better health estimation, longer life span, more reliable range prediction and further benefits~\cite{calif_proper_2008}.

%
%
The development of effective methods for State-of-Charge (SoC) estimation remains a topic of crucial research focus.
Various techniques to estimate the SoC have been developed to enhance battery usage.
The ability to determine the state of a battery or a battery system is a required function for an advanced Battery Management System (BMS).
Those techniques can be classified into three primary categories~\cite{ali_towards_2019,ng_enhanced_2009,robust_SoC,6953745}: direct measurement, model-based methods, and computer intelligence or Machine Learning (ML).
Direct measurements methods take readings from batteries relying on sensors, such as open circuit Voltage, internal resistance, or current readings over set periods (i.e. Coulomb Counting)~\cite{ng_enhanced_2009,robust_SoC}.
Model-based methods recreate battery behaviour and use sensor inputs to calculate results from a pre-defined model~\cite{6953745}.
Computer intelligence techniques enhance such models with additional data.
Those data-driven calculations aim to improve model estimation by fitting to an actual behaviour observation.
Examples include Fuzzy Logic~\cite{malkhandi_fuzzy_2006}, Support Vector Machine~\cite{hansen_support_2005, anton_battery_2013}, or Neural Networks (NN)~\cite{song_lithium-ion_2018,Chemali2017,mamo_long_2020,jiao_gru-rnn_2020,xiao_accurate_2019,javid_adaptive_2020,zhang_deep_2020}.

%
%
While some model-based methods, such as the equivalent circuit model, are simple to implement within a BMS, many cannot correctly capture batteries' complex multi-dependant behaviour~\cite{6953745}.
%Others are limited to offline simulation due to high complexity and computation requirements, therefore not suited to an onboard BMS.
Direct measurement estimation is limited to sensor accuracy and affected by losses created by Coulombic efficiencies~\cite{Smith_2010} where some portion of charge gets transferred to heat or gets affected by battery ageing that is not captured.
%Additionally, the measured voltage depends on whether the battery is in use or has been on rest for a given time.
%An accurate SoC estimation model has to implement a way to accommodate battery losses and sensor readings.
%If model-based methods treat as constant values, then in real scenarios, they can be significantly limited to sensor inaccuracy, environmental and internal battery temperature, or aging.
%In contrast, Neural Network methods can accommodate losses as they can capture complex phenomenological behaviours~\cite{bengio_learning_1994}.
%
%Smart BMSs, which incorporate some method of ML, use only sensory data~\cite{zhang_deep_2020} and do not require a batteries physical property~\cite{zhang_deep_2020}.
In contrast, Machine Learning are known to be able to establish relationships in complicated and multi-dimensional non-linear systems~\cite{hansen_support_2005,anton_battery_2013,he_state_2014}.
This characteristic shows great potential to account for battery losses due to Coulumbic efficiency.
Some researchers used the support vector machine-based methods to estimate SoC using voltage, current, and temperature inputs~\cite{hansen_support_2005,anton_battery_2013}.
Data were obtained from a driving schedule profile on a battery cycler, and the end error estimation was achieved as less than 6\%~\cite{he_state_2014}.
%The solution to lowering that percentage may lie within the more complicated models, like Neural Networks.
Many attempts to implement different Neural Networks exist, but the most promising net for charge estimation were Recurrent Neural Networks (RNN)~\cite{song_lithium-ion_2018,Chemali2017,mamo_long_2020,jiao_gru-rnn_2020,xiao_accurate_2019,javid_adaptive_2020,zhang_deep_2020}.
Their effectiveness in time-series dependant problems was shown using internal neurons to process data sequences with varying lengths by Chemali et al.~\cite{Chemali2017}.
% The models' cells act as memory units, building relationships and giving outputs based on multiple inputs over time.
% Typical examples are data forecasting, handwriting, speech and image recognition, machine translation or music composition~\cite{devdarshan_applications_2019}. 

% 
%
Over the past years, the RNN approach has found multiple applications for SoC estimation.
The most recent attempt to determine the Li-Ion battery's remaining useful life was with implementation of the Gradient Recurrent Unit models~\cite{song_lithium-ion_2018,javid_adaptive_2020,xiao_accurate_2019,jiao_gru-rnn_2020}.
The earliest approach utilised a battery charge's regression nature only using stateless models~\cite{song_lithium-ion_2018,jiao_gru-rnn_2020,xiao_accurate_2019} when input impact is not preserved for further predictions.
Later, some approaches introduced additional parameters to support the NN learning process~\cite{mamo_long_2020,jiao_gru-rnn_2020,javid_adaptive_2020}.
Besides good convergences, the model's use was in determining critical events, like the time before complete charge depletion or overcharge.
However, it did not receive wide application due to its initial state requirement as input features.
The most popular way lies in determining the charge's value using a fixed size history of voltage, current, and temperature in stateless Long Short-Term Memory (LSTM) models~\cite{Chemali2017,mamo_long_2020,javid_adaptive_2020,zhang_deep_2020}.
The method's advantage is independent of the charge or discharge cycles at different times, as long as the history samples are in order.
However, estimation determines values based on fixed history samples and does not preserve for the next prediction, making practical estimation of single or several charge values, but not determining critical events ahead.
That can be achieved with stateful models, which continuously preserve the prediction's impact and can be propagated further until they reach the end state without resets~\cite{zhu_statefulnes_tfdocs_2020}.

%
%
One of the first attempts to train an RNN model to predict the state of charge is to train in several cycles of a single battery utilisation dataset at different temperatures~\cite{song_lithium-ion_2018, xiao_accurate_2019,javid_adaptive_2020, jiao_gru-rnn_2020}.
Then as others attempted to generalise battery behaviour to multiple usage scenarios, which as comparison lead to higher Root Mean Squared Error.~\cite{mamo_long_2020}.
For instance, by comparing similar procedures of testing from Song et al.~\cite{song_lithium-ion_2018} and Mamo et al.~\cite{mamo_long_2020} who conducted their research methods but used different validation mechanisms, it can be seen how accuracy error doubles if testing performs at a different temperature or untrained driving profile.
By selecting same drive cycles at closely similar temperature, Song et al.~\cite{song_lithium-ion_2018} accuracy acchieved rougly 0.735\% error, then as Mamo et al.~\cite{mamo_long_2020} reported 1.2533\% at best.
This can be explained that most researchers used a single driving profile, but with an entire available temperature range for training and then a portion of handpicked temperatures for validation and accuracy report.
This behaviour is less realistic to the actual scenario since during a single acceleration event, the battery goes through ambient temperature to the maximum allowed within several seconds, and its' usage depends on the condition of a road and drivers experience.
Mamo et al.~\cite{mamo_long_2020} attempted to generalise this behaviour by using the addition layer construction technique to make a model be appliable to any driving condition and more realistic scenario.
There has been little research to validate the performance of different Machine Learning techniques to extrapolate ideal laboratory batter cycling conditions of early collected data to an electric vehicle behaviour.
%% (Include to one of the sections)
% After identifying the most optimal input preferences without significant effect on the performance, one of the further developments apply changes to the structure of GRU or LSTM by adding additional layers (Attention or Extra Dense layers)~\cite{mamo_long_2020, jiao_gru-rnn_2020}.
% Another way is to change the optimisation process to achieve similar accuracy faster (i. e., adding a Momentum algorithm to the Stochastic Gradient Optimisations process or replacing gradient calculations with statistical)~\cite{xiao_accurate_2019, javid_adaptive_2020}.
% Those methods modify standard ways introduced earlier in model training by applying additional operations.
% As a result, they met to achieve better accuracy faster using similar training approaches. 
However, after identifying the most optimal input preferences, some researchers applied changes to the structure of models by adding additional layers (Attention or Extra Dense layers~\cite{mamo_long_2020, jiao_gru-rnn_2020}) or modified optimisers (i. e., adding a Momentum algorithm to the Stochastic Gradient Optimisations process~\cite{xiao_accurate_2019}) to further enhance the speed of the training process to achieve the lowest error.
\mbox{Table~\ref{tab:review}} summarises the most common methods applied to SoC estimation, highlighting the model cell type, structure to define input sample type, optimiser selection, and additional features introduced by authors to improve predictions.
\textcolor{red}{\textbf{Matt: I moved most of the details about the table to caption. However, since this paragraph and next were swapped I had to do my best to avoid driving profile mentioning where possible. Perhaps we should consider swapping them back or some sentences.}}
\begin{table*}[h]
    \renewcommand{\arraystretch}{1.3}
    \caption{Reviewed papers implementation summary.
    The model type highlights a primary path in structuring a Neural Network.
    The statefulness defines the input method, where stateless uses a fixed size of input samples per each feature and statefully apply each time sample at a time for all features one by one.
    %Optimiser selection sets the algorithm for the learning process from one of the following methods: Adaptive moment estimation (Adam), Nesterov adaptive moment estimation  (Nadam), Stochastic gradient descent (SGD), AdaMax (AM) and Differential Evolution (DE).
    Optimiser defined from: Adaptive moment estimation (Adam), Nesterov adaptive moment estimation  (Nadam), Stochastic gradient descent (SGD), AdaMax (AM) and Differential Evolution (DE).
    }
    \centering
    \label{tab:review}
\resizebox{\textwidth}{!}{
\begin{tabular}{l|c|c|c|c|c|c|c|c|c|l}
    \hline\hline \\[-4mm]
    \multicolumn{1}{ c }{} & 
    \multicolumn{2}{|c|}{Model} & 
    \multicolumn{2}{ c|}{State-} & 
    \multicolumn{5}{ c|}{Optimiser} &
    \multirow{2}{ 4em }{Extension} \\
    \cline{1-4} \cline{5-10}
    Reference source & GRU  & LSTM & -less & -ful & Adam & Nadam & SGD & AdaMax & DE\footnote{Differential Evolution} & \\
    \hline
    Song~\cite{song_lithium-ion_2018}
        & \chk &      &      & \chk & \chk &      &      &      &      & 4 Layers  \\
    Chemali~\cite{Chemali2017}
        &      & \chk & \chk &      & \chk &      &      &      &      &           \\
    Mamo~\cite{mamo_long_2020}
        &      & \chk & \chk &      &      &      &      &      & \chk & Attention \\
    Jiao~\cite{jiao_gru-rnn_2020}
        & \chk &      &      & \chk &      &      & \chk &      &      & Momentum  \\
    Xiao~\cite{xiao_accurate_2019}
        & \chk &      &      & \chk &      & \chk &      & \chk &      & Ensemble  \\
    Javid~\cite{javid_adaptive_2020}
        & \chk &      & \chk &      & \chk &      &      &      &      & Robust    \\
    Zhang~\cite{zhang_deep_2020}
        &      & \chk & \chk &      &      & \chk &      &      &      & Online    \\
    \hline\hline
\end{tabular}
}
\end{table*}

% Althpugh, experement which involved extrapolation ideal laboratory conditions with accurate sensory to a real world conditions, woth different environmental charecteristic or sensorary interferance - is not currently available.
%
In most published testing of ML methods applied to SoC, experiments on battery cycling data were conducted on different cell types.
Most used table data of real-time sensory results from battery cyclers to validate efficiencies generated using different current schedulers (driving profiles)~\cite{Chemali2017,song_lithium-ion_2018,mamo_long_2020,jiao_gru-rnn_2020,xiao_accurate_2019}.
%An equally time-based sample of current consumption acts as an input to the battery cycler, intended to recreate a stress test on a battery or human driving behaviour.
There are three the most commonly used in the research of this area: Dynamic Stress Test (DST) for a variable power discharge mode, aggressive Highway Drive Schedule (US06) and Federal-Urban Driving Scheduler (FUDS) for nominal driving scenarios.
Unlike some general simple static discharge processes, which commonly appear in other battery-based tools, driving profiles include some amount of regenerative driving to simulate the actual application of the battery for an electric vehicle.
%
% Mamo et al.~\cite{mamo_long_2020} conducted experiments by validating the RNN model performance by training one driving profile and testing against the other two.
% The results showed doubled higher Root Mean Squared Error, as opposed to training and validation over single driving profiles as per experiments by Xiao et al.~\cite{xiao_accurate_2019}.
%
The process of identifying the best-suited method for a specific condition, like driving an EV, is one of the crucial steps for machine learning engineering.
It requires carefully defined methodology, which characterises researched conditions as close as possible, and experimental results from multiple models with appliable techniques and the lowest errors.
By comparing implementation and results from different sources and testing accuracy and performance against multiple driving conditions at various temperatures from ambient to maximum possible, it is possible to select the best machine learning technique, which can be integrated directly into an Electric vehicle and safely used either on tight city roads or long high-speed highways.
%There have not been many similar experiments against other SoC estimation methods since such validation procedures are necessary for the computer intelligence method compared to battery modelling.
% In addition, all investigations did poor research in extrapolating ideal laboratory conditions with battery cycler, stable temperatures and no sensor discrepancies to an actual driving situation on the road, mainly due to lack of car data.
%
% Usage of a single profile's battery cycling data may not validate ML methods' efficiency in driving an electric vehicle.
% The inability to determine the charge's current state during EV driving makes the online learning process inapplicable.
% Even with other SoC estimation technique usage, the computational complexity of training any NN is complicated to fit on a low power device.
% An offline trained model had the advantage of insignificant resource consumption during the prediction stage, making it a prefered way for an EV.
% All further model testing will be applied through varying battery cycling profiles to capture the influence of data type to model efficiency.
%
% (No point)
% The drawback lies within the validation procedures of produced models.
% Ideally, this process must be performed with similar data but under different conditions or at different times.
% However, once the produced model is taken through different unseen scenarios, the accuracy lowers drastically compared to early reported results since it starts to experience something completely unexpected.
% The lack of training samples or means to produce at different conditions affected the final judgment results of many publications. 
% Even though the estimation produces accurate output with tabled data, placed under actual driving conditions - the model will have difficulties matching training performance. 


%
%
% Mamo et al. conducted experiments
%
% ------------------------------------------------------------------------ \\
% PAPER CONTRIBUTION may be worth adding as a clear statement \\
% ------------------------------------------------------------------------ \\
% Asses handpicked articles from different categories of SoC estimation, asses efficiency and complexity to utilise in EV.
% This paper's contribution lies in researching recent approaches used to estimate Charge's State from battery sensor readings.
% Comparing and implementing the most promising algorithms intended to determine the direction to build upon in the future. \\ \\
% This paper will test recent developments in the most common SoC prediction to determine the most promising direction/approach towards implementing an improved and light model applied to an electric vehicle.
% It determined to find the ML model creation's weakness and implement a new one for an EV. \\
% ------------------------------------------------------------------------ \\
%
%
% This paper investigates, implements and compares extended memory-based models of RNN to predict the State of Charge and additional built-on over time techniques to select the most effective and least resource requiring onboard-based computations appliable within Electrical Vehicles.
This paper investigates, implements and compares extended memory-based models of RNN to predict the State of Charge and additional built-on over time techniques to select the most suitable for practical application to be used in EV using combinations of different profiles.
Each subset will contain implementation from various articles changing either structure of the models or learning approaches.
It should help develop a methodology that will be further used to extrapolate offline trained methods from the lab condition to the road drive tests.
% The recent advancement and commonly used subsets are Gradient Recurrent Unit and Long Shot-Term Memory unit cells.
% Recurrent Neural Network has been confirmed to be suitable for the battery-related system by authors discussed further, such as Chemali~\cite{LSTM_Hochreiter1997}.
% However, there is no valid proof of if GRU or LSTM is helpful for battery SoC estimation.
% The best way to separate them and explore differences and efficiencies is to use them as Stateful and Stateless models.
The A123 Lithium-Ion battery data with three typical driving profiles, obtained from the University of Maryland 2012~\cite{noauthor_calce_2017} cycling experiment will act as training and testing samples.
Each method will be validated through those samples (either DST, US06 or FUDS driving profiles) and tested against robustness and accuracy of estimation of State of Charge of batteries in the other two unseen schedulers.
%
The remaining sections are organised as follows: all methodology with details regarding training procedure outlined in Section~\ref{sec:Meth}, with processing data in Subsection~\ref{subsec:b_data} and testing hardware in Subsection~\ref{subsec:soft}.
Every detail on algorithms and optimisers are written in Section~\ref{sec:Body}, where Subsection~\ref{subsec:structure} separates all details for each GRU and LSTM method, and Subsection~\ref{subsec:optimisers} breaks down every applied optimising algorithm.
Section~\ref{sec:conclussion} gives the results of implementation, performance characteristics and concludes the critical analysis.

%%% Body
\section{Recurrent Neural Network (RNN)}\label{sec:RNN}
Implementation of the model based on simple RNN using multi layer Dense(X) networks.~\cite{lees2010theoretical}
\section{Long-Short Term Memory (LSTM)}\label{sec:LSTM}
Implementation of the model based on Chemali2017. Application for our section and results. Refer to methodology from time to time.
\section{Gradient Recurrent Unit (GRU)}\label{sec:GRU}
Implementation of the model based on BinXio or someone else. Introduce their way of implementation how to improve optimisation using 2 algorithms as per their discussion.
\section{LSTM with Attention}\label{sec:lstm-attention}
Implementation of the model based on TadeleMamo 2020 as the most recent and how to implement improvement to the model to make it better. Gracefully transition idea from here to my model, it is similar.
%%% Conclusion
Since it is a common practice for temperatures on the battery inside EV to spike from 20 ambient degrees to the limit of 60, all temperature ranges were used together to train each model.
The training process was conducted through all datasets for a single battery testing profile and validated on a single cycle of unseen data of 25\textdegree{}C (less or around 20\% of the entire set).
This approach led to the accuracy being lower than other researchers reported by training individually for temp ranges like with Xiao et al.~\cite{xiao_accurate_2019}.
The following section compares the models trained on each individual and then tested against the entire dataset of all three profiles but of a different cell.
All stateless examples were trained using charge and discharge cycles unlike stateful models.

%
%
% Final tests for a model performance were conducted against an entire set of two remaining profiles separately.
The metrics were reported using the equations outlined in the \mbox{Table~\ref{tab:metrics}}.
Figures were generated during each iteration of the training process from the data samples outlined in the \mbox{Subsection~\ref{subsec:b_data}}.
After completing the predefined amount of epochs, each metric was recorded in a comma-separated file to produce accuracy plots, allowing to assess the efficiency of the learning process.

%
%
\mbox{Two tables, \ref{tab:acc-results1} and \ref{tab:acc-results2}}, contains results of accuracy validation on six implemented models over entire drive cycles datasets.
\mbox{Figures between \ref{fig:Model-1res} and \ref{fig:Model-6res}} demonstrated the best-selected cases for visual demonstration and comparison of training on one profile and validation against the other two.
\mbox{Figures \ref{fig:Model-1losses} to \ref{fig:Model-6losses}} refer to the best model over the learning process, based on minimum Mean average or Root Mean Squared errors.

%
%
\subsection{Models results overview}
Implementation of several different variations of time-series modells allowed to analysise multiple path of evolution of machine learning techniques in State of Charge estimation in the past three years.
Review of the resulted accuracy helps make a reasonable justification for further research.

%
%
Model \#1 has been based on the most simple and oldest research from 2017, made by Chemali \textit{et al.}~\cite{Chemali2017}.
It utilised the simplest model structure with a single layer and no complicated cell structure or optimiser manipulation.
By the results of the general accuracy overview at Table~\ref{tab:acc-results1}, it has the most stable accuracy results across three different profiles in comparison to the models.
It can be justified by the longest training time to reach the lowest error.
This model has shown good accuracies on US06 and FUDS datasets, but the training has not been very smooth, unlike with DST, as per Figure~\ref{fig:Model-1losses}.
By analysing Figure~\ref{fig:Model-1res}, the DST had the worst results in capturing general behaviour. Subfigure~\ref{subfig:Model-1res-DSTvsFUDS} illustrates the area plot, with the highest peaks in comparison to other plots.
\textbf{Capturing the middle area of charge, where voltage remains stable most of the time, is not very easy, and DST may not be the best model for it.}
Contrary to DST, the US06 made the best result in capturing both features.
By analysing Table~\ref{tab:acc-results1}, US06 and FUDS results for Model \#1 can capture each other behaviour, but not the DST.

%
%
Model \#2 was an attempt to move from an old LSTM to a recently developed GPU type of cell.
Measuring the impact of such migration is not feasible.
However, it utilised two different optimisers for pre-tuning and fine-tuning phases.
The transition happened after a third of the maximum allowed iterations, which in some cases lead to high accuracy spikes on the training process, seen in Figure~\ref{fig:Model-2losses}.
The pre-tuning phase might have decreased the time to achieve the optimal results, but considering the number of input samples used - the difference is not noticeable between the LSTM and GPU types of models.
Lowering the learning rate and switching the optimiser led to a much stable learning curve but did not bring any improved prediction results.
Despite that training went smoother, except for minor anomalies on the FUDS training as per Model~\ref{fig:Model-2res}, the overall behaviour capture across profiles did not improve.
\textcolor{red}{Matt:DO I even need to share with my thoughts and refer to Table with potential why results became worse? I don't have an answer myself yet}

%
%
Adding the attention layer as per Model \# 3 did not make an overall improvement system improvement, but it produced much smoother outputs.
The prediction is minor variant, more expectable behaviour referring to a state of charge.
The less fluctuation model produces in the SoC prediction, the better usability it has.
Model \#3 was able to achieve the lowest error twice faster but also went to overfit without learning step adjustment, as per Figure~\ref{fig:Model-3losses}.
It could make a reasonably good capture of all three profiles with US06 profiles.
Then as the FUDS model became much smoother in training with the attention layer, but still has difficulty in capturing 50\% charge in the testing scenarios on subfigure~\ref{subfig:Model-3res-FUDSvsUS}. 
% made an improvement on the FUDS training and better fir on Figure 9.i.
% Although, it had promessing start by the training accuracyy, the technique of adding more modefied functional has a means to make model better rather than just adding more layers to the model.
% Instable, variet, noisy

%
%
Unlike other methods, Model \#4 utilised a stateful technique for the data input pipeline.
Despite that, it has a promising approach but only in specific scenarios.
For example, the training over DST has been conducted with both charge and discharge together, resetting the model at the end of the cycle, as per subfigure~\ref{subfig:Model-4res-DSTvsDST}.
Having a charge in the system added complexity in the capture, but the error had a stable degradation over time as per Figure~\ref{fig:Model-4losses}.
However, it had to be cut off early, since training threshold has not been reached in the set limit.
It was attempted to be fixed on US06 and FUDS based models, applying only the discharge process.
However, it did not bring much of the improvements and only made charge prediction more accurate comparedto DST.
For the general behaviour capture, stateful models are not suited by themself only.
However, training two models separately for both charge and discharge or use it additional techniques on short prediction, for example, during acceleration events - the stateful model has usable applications.

%
%
Similarly to the second method, Model \#5 applied a different optimiser.
However, additional complexity to other processing introduced better overall accuracy Table~\ref{tab:acc-results2}, as reported by Javid \textit{et al.}~\cite{javid_adaptive_2020}.
All three models experienced better training convergence in a shorter time, as per Figure~\ref{}.
However, the difficulty in capturing middle of a charge is still preserved.
% Model 5: 1,3 mdeol applied modification to the structure directly and 4th one indirectly. (2nd one did not, read again).
% There as (2?) and 5 improves optimisation steps, leading to better accuracy in all testing.
% RoAdam allowed faster convergences to the optimal accuracy and generaly achioeved better results by Table 2.

%
%
Model \#6 applied an additional LSTM layer with separated neurons to test better capturing.
However, the result did not bring better error or faster convergence.
In some cases, like wi\mbox{Two tables, \ref{tab:acc-results1} and \ref{tab:acc-results2}}, contains results of accuracy validation on six implemented models over entire drive cycles datasets.
\mbox{Figures between \ref{fig:Model-1res} and \ref{fig:Model-6res}} demonstrated the best-selected cases for visual demonstration and comparison of training on one profile and validation against the other two.
\mbox{Figures \ref{fig:Model-1losses} to \ref{fig:Model-6losses}} refer to the best model over the learning process, based on minimum Mean average or Root Mean Squared errors.

%
%
\subsection{Discussion to add}
After analysing all six models, there are several comparisons, which can be derived from them.
Model \#1 and \#6 have tested two approaches of neuron handling, either at single or dividing between multiple layers.
As a result, there was no significant advantage unless there were more modifications to the multilayer sequential model.
There as Model \#3 introduced additional custom computation without extra memory cells, leading to a smoother output.

%
%
Model \#2 and \#5 approach the convergence with optimisers modification.
The ensemble used two algorithms to speed up the long process of locating local minimum and the second algorithm to tune up closer to it with an adjusted learning step.
However, the modified Adam version with a direct impact of the loss function improved the result drastically.
The performance profiling has not beeing conducted to asses how kfsdj;kfj; blin I am exossed!!!!1 AAAAA

%
%! [Refer to the bottom dicussion of stateful model]
Model \#4 was the only one, which used stateful model for training and testing.
Since, every time-series model is stateful internally, the difference is in the moment of the model reset.
It is more convinient for State of CHarge scenarion to work with short burst of data, rather than attempt to preserve very long dependancies in a single run, and be limited to initial conditions.

%
%
Generally DST is bad for capturing as shown on all plots, which was expected initially. Although there was a chance that Dynamic stress test may act as a middle ground between Urban and highway driving.
Acubs as comparison to model 1.
However, the US06 acted as a better model, due to AAA!!!!

Separating to multiple layers to the same amount of untis did not lead to improvements.
It was the fastest what achieved the lowest training accuracy, but without affecting learning rate, model could not achieve capturing the other trens.

%
%
This all shown that model not able to capture middle area.
However, combining all 3 techniques into single model may lead to accurate results.

%
%
Model 1,5,6 may be matter of randomness, more that just technique eficiency.
DST generally faster singe profile itsekf is very easy to handle.
+ Model 4 had no means to perform test on entire set of both profiles.
%%%%%%%%%%%%%%%%%%%%
%% PUT both writing and the results from. After you do thatm compare the results as discussion. Expplicitly interpreter that.
%
%
% \subsection{******}
%
%
Despite multiple tests over two cell types, there is no apparent advantage in using LSTM or GRU layers.
To determine the actual performance, it will require multiple trained models over a single implementation to average performance results and make a clear statement.
Models \#1 and \#2 give an illusion of an advantage to one model over another.
However, the accuracy plots in \mbox{Figures~\ref{fig:Model-1losses} and ~\ref{fig:Model-2losses}} indicate how error degrades with time for both models.
The advantage of one over another is a simple matter of randomness in the initial training results.
The attention layer may not significantly boost the training or accuracy, but it gives a good foundation for further improvements and modifications.
Since the SoC estimation is not a pure number based behaviour but also a matter of physical, electrical properties, manual adjustment weights, losses or data itself will not bring valuable results.
Further research and adjustments must be made using a similar principle to improve the training procedure with augmented models. 
For example, the sigmoid function selection in the model output minimises the possibility of going over 100\% or 0\% of charge.

%
%
The best performance with Stateful models can be achieved through using a separate training process for charge and discharge. 
\mbox{Subfigure~\ref{subfig:Model-4res-DSTvsDST}} demonstrates training over DST, which sufferers from high error in both charge and discharge process.
The other two training were performed with discharge sets only, \mbox{Subfigures~\ref{subfig:Model-4res-UStr}, \ref{subfig:Model-4res-FUDStr}}.
Stateful models can not be validated using traditional means of accurate measurement.
\mbox{Table~\ref{tab:acc-results2}} for Model \#4 takes results from a single cycle only, same as~\ref{fig:Model-4res}, since there is no straightforward way to validate across all cycles (marked with $*$ symbol).

%
%
The advantage of modifying optimisers is better observed on the accuracy plots, \mbox{Figure~\ref{fig:Model-2losses} and ~\ref{fig:Model-5losses}}.
Model \#2 had better results in avoiding overfitting since it used two optimisers for quick adjustment and tuning.
Similar to Model \#5, which had a small learning rate, but modified parameter update with direct involvement of the loss values.
An early termination over \#5 and \#6 is a result of overfitting or apparent stability in the accuracy.
With the number of samples, which the training process went through and based on the RMSE plot, there was a little need for repeated training over the same data repeatedly, as proven in the first several models.

%
%
Overall, there is an obvious advantage in training a model over a single profile and then testing against similar scenarios, for example, creating a model that fits a single drive's driving behaviour over specific driving scenarios.
However, it will suffer from inaccuracies if models are placed under other conditions without a post-training process with new data.
Comparison of the validation data act as a determined prove.

The work offered several implementations of Machine Learning algorithms for State of Charge estimation for A123 batteries. \\ [2pc]

%
%
\textbf{implementation of the model based on TadeleMamo 2020 as the most recent and how to implement improvement to the model to make it better. Gracefully transition an idea from here to my model. It is similar.}

%
%
Even though most of the model provided excellent results, they all lacked the accuracy of the Time-series model in other similar predictions. The reason for that is the weights, which the model places on the features and lacks State of Charge reading as input. 
\textbf{The usage of any training profiles for real electrical vehicles falls into a limitation of temperature data. The model tends to put equal weight on the Temp feature. Considering that all values within a tolerance of 1 degree, the actual car will increase or decrease slope over utilisation. All model has proven to capture complicated behaviours as per FUDS schedule, but not very efficient in the general characterisation of the system. That fact made DST acts as a better learning input ...}
\textbf{Most of the time, the model placed most of its' weight on Voltage readings, then current or Amp/hours make a more significant impact on a battery cell. Voltage makes it the most obvious one.}

%
%
In the second case, the property of the predicting output makes it questionable to use.
On the one hand, using simply Coulumb Counter may fix this problem, then a model may have enough information to predict the total battery consumption.
On the other, due to the Coulumbinc efficiency of Li-ion cells, CC methods suffer from many factors that make reading inaccurate.
By training the model on the perfect state of charge from battery testers and then using CC reading from battery utilisation within a device, Machine learning along with other methods such as Kalman filter may prove itself more effective****.
\textbf{All the methods suffer from one thing, having SoC as an input feature. There are two ways to deal with either a perfect sensor or a good model that does not mess up itself when applying results.} \\
\textbf{The Chip-on-board devices, which has potential usage inside an Electrical vehicle, are limited to Stateless model implementation. The most computationally effective approach has been measured over two potential devices.}

%
%
The model training process consists of N essential steps.
The methods described in this article applied different methods to all three independently.
This articles summarised all of them together to determine the reasonable approach for further improvements.
From the results, the choice of optimiser affects the speed of the training offline but closely helps achieve better performance in the online one.
However, all models lacked additional input features as SoC.
The methods of whether or stock price forecasting.
The inability to accurately estimate SoC and the current time and then using it as an input for further predictions reduces chances for good estimation.
The accuracy of other SoC estimation methods, like CC, prove themself ineffective due to inaccuracies. 
To achieve that, the most reasonable approach is to use the approach by Tadele Momo et al.~\cite{mamo_long_2020} and modify the structure of the model to accommodate SoC not as just history dependant, but ....
The implementation of this method and further investigation is the topic for another article.
\section{Acknowledgements}\label{sec:acknowledgements}
The research was undertaken through funding from the Automotive Engineering Graduate Program (AEGP) in cooperation with the Queensland government and Prohelion industry partner.
All models evaluation has been performed through the help of the QUT HDR research and technical staff, who arranged access to the Hight Performance Machine (HPC Lyra) for the extensive initial computations.
The research has been conducted in cooperation with Mr Sam Haines, who performed a similar investigation on a similar dataset using Kalman filters and its' branches.
In addition, both Sam and I owe our progress to our Principal and Associate supervisors, Associate Professors David Holmes and Geoff Walker, who have been following our progress through the entire research.
Finally, we gratefully acknowledge the original developers of the Machine Learning framework Tensorflow from Google, who wrote detailed guides and documentation for all necessary tools used throughout the investigation.
%It is worth acknowledging the input from Dr Olga, Dr Mahsa and Pr Lovell who showed a practical interesr in the research and ... contribtution.



% \section*{Methods and Materials}

% Guidelines can be included for standard research article sections, such as this one.

% \section*{Some \LaTeX{} Examples}
% \label{sec:examples}

% Use section and subsection commands to organize your document. \LaTeX{} handles all the formatting and numbering automatically. Use ref and label commands for cross-references.

% \subsection*{Figures and Tables}
    
%     Use the table and tabular commands for basic tables --- see Table~\ref{tab:widgets}, for example. You can upload a figure (JPEG, PNG or PDF) using the project menu. To include it in your document, use the includegraphics command as in the code for Figure~\ref{fig:view} below.
    
%     % \begin{figure}[ht]
%     %     \centering
%     %     \includegraphics[width=0.7\linewidth]{frog}
%     %     \caption{An example image of a frog.}
%     %     \label{fig:view}
%     % \end{figure}
    
%     \begin{table}[ht]
%         \centering
%         \begin{tabular}{l|r}
%             Item & Quantity \\\hline
%             Candles & 4 \\
%             Fork handles & ?  
%         \end{tabular}
%         \caption{\label{tab:widgets}An example table.}
%     \end{table}

% \subsection*{Citations}

%     LaTeX formats citations and references automatically using the bibliography records in your .bib file, which you can edit via the project menu. Use the cite command for an inline citation, like \cite{lees2010theoretical}, and the citep command for a citation in parentheses \citep{lees2010theoretical}.

% \subsection*{Mathematics}

%     \LaTeX{} is great at typesetting mathematics. Let $X_1, X_2, \ldots, X_n$ be a sequence of independent and identically distributed random variables with $\text{E}[X_i] = \mu$ and $\text{Var}[X_i] = \sigma^2 < \infty$, and let
%     $$S_n = \frac{X_1 + X_2 + \cdots + X_n}{n}
%           = \frac{1}{n}\sum_{i}^{n} X_i$$
%     denote their mean. Then as $n$ approaches infinity, the random variables $\sqrt{n}(S_n - \mu)$ converge in distribution to a normal $\mathcal{N}(0, \sigma^2)$.

% \subsection*{Lists}

%     You can make lists with automatic numbering \dots

% \begin{enumerate}[noitemsep] 
%     \item Like this,
%     \item and like this.
% \end{enumerate}
% \dots or bullet points \dots
% \begin{itemize}[noitemsep] 
%     \item Like this,
%     \item and like this.
% \end{itemize}
% \dots or with words and descriptions \dots
% \begin{description}
%     \item[Word] Definition
%     \item[Concept] Explanation
%     \item[Idea] Text
% \end{description}

% \section*{Acknowledgments}

%     Additional information can be given in the template, such as to not include funder information in the acknowledgments section.

\bibliography{sample}

\end{document}